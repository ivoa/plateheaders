\documentclass[11pt]{ivoa}
% We need a bit of extra width for 80 tt characters/line
\usepackage[textwidth=14cm,a4paper]{geometry}
\usepackage{longtable}
\input tthdefs
\lstset{basicstyle=\ttfamily\footnotesize,basewidth=4.8pt}
\usepackage{todonotes}

\definecolor{keyword}{rgb}{0.118,0.2,0.618}
\newcommand\cardname[1]{\texttt{\color{keyword}#1}}

\title{FITS Headers for Scans of Photographic Plates}

\ivoagroup[IG]{Data Curation and Preservation}

\author{Demleitner, M.}
\author{Tuvikene, T.}
\author{Schmalz, S.}
\author{Enke, H.}
\author{Partl, A.}

\editor{Demleitner, M.}

% origin: https://www.plate-archive.org/applause/wiki/fits-header-format-dr2/
% \previousversion[????URL????]{????Concise Document Label????}
\previousversion{This is the first public release}

\newenvironment{fitsexample}[1]
{\bigskip\noindent\textbf{Example}\\\textit{(#1)\smallskip}}
{\medskip}

\begin{document}
\begin{abstract}
This note describes a FITS header structure suitable for the annotation
of scans of photographic plates containing sky images intended for
scientific use.  This metadata structure is general enough to cover
several observational techniques common in the plate area, such as
multiply exposed plates or objective prism spectroscopy.  While this
metadata structure was being developed together with annotation software
(PyPlate) and a specific digitisation project
(APPLAUSE), we hope it is general enough to serve as a lingua franca for
the annotation of scans of historical photographic observations.

\end{abstract}


\section*{Conformance-related definitions}

The words ``MUST'', ``SHALL'', ``SHOULD'', ``MAY'', ``RECOMMENDED'', and
``OPTIONAL'' (in upper or lower case) used in this document are to be
interpreted as described in IETF standard RFC2119 \citep{std:RFC2119}.

The \emph{Virtual Observatory (VO)} is a
general term for a collection of federated resources that can be used
to conduct astronomical research, education, and outreach.
The \href{https://www.ivoa.net}{International
Virtual Observatory Alliance (IVOA)} is a global
collaboration of separately funded projects to develop standards and
infrastructure that enable VO applications.


\section{Introduction}

This document  proposes FITS \citep{std:FITS} header keywords, value
types and an overall layout for storing metadata of digitized
astronomical photographic plates, where we use ``plates'' here to
include images on both glass or film.

The header format is intended to suit
various cases: direct images with single exposures, multiple exposures
of a single object/field, exposures of different objects/fields,
objective-prism spectra, etc.

The goals of the present work is to 

\begin{itemize}
\item facilitate writing software for automatically processing such data
given the constraints of the historical material,

\item aid data providers in selecting useful and relevant metadata to
add to their scans,

\item provide researchers with easily-inspectable information not only
on the image itself, but also on its provenance.
\end{itemize}

The present work has greatly benefited from the Wide-Field Plate
Database WFPDB \citep{1995LNP...454..412T}, which informed many of the
design choices made here.  However, this effort is independent of the
WFPDB as such.  The direct predecessor of this Note is \citet{paper1},
which already introduced several of the concepts covered here.

While conformant FITS headers can be written with generic FITS software,
higher-level interfaces have been written, in particular in the PyPlate
software for calibrating and annotating scans of astronomical plates
\citep{soft:pyplates}, but also in the general Virtual Observatory
server suite DaCHS \citep{2014A+C.....7...27D}.

This specification's core is Section~\ref{sect:format}, which describes
the various groups contributing to a full description of a scanned
plate's metadata.  A complete
sample header is shown in Section~\ref{sect:samplehdr}.

\section{FITS header format}
\label{sect:format}

For better readability of the header, we organise keyword records
into groups of related keywords and separate these groups with
cards that have a blank keyword:

\begin{lstlisting}
KEYWORD1= 'value   '           / sample keyword
KEYWORD2= 'value   '           / sample keyword
        --------------------------------------- Original data of the observation
KEYWORD3= 'value   '           / sample keyword
KEYWORD4= 'value   '           / sample keyword
        ----------------------------------------------------- Photographic plate
KEYWORD5= 'value   '           / sample keyword
KEYWORD6= 'value   '           / sample keyword
        --------------------------------------- Computed data of the observation
KEYWORD7= 'value   '           / sample keyword
KEYWORD8= 'value   '           / sample keyword
\end{lstlisting}

An italic $n$ in header keywords denotes numbering, where $n$
can range between 1 and 99.  \cardname{RA$n$} will thus expand to
keywords \cardname{RA1}, \cardname{RA2}, \cardname{RA3}, etc. Numbers in
keywords are not padded with zeros.

All headers defined here are optional in the sense that a header without
them still conforms to this specification.  In particular, not even the
group headers are mandatory.  However, \emph{if} a header card is present
with a certain keyword defined here, that card must conform to the
syntax and semantics defined here to be compliant.

In addition to the requirements and recommendations given here, all
constraints resulting from the various standard documents on FITS apply.

This specification does not use any of the FITS conventions on long
keywords.  However, FITS CONTINUE
cards\footnote{\url{https://fits.gsfc.nasa.gov/registry/continue_keyword.html}}
are explicitly
allowed (and encouraged for some of the free text fields).  This should
no longer cause interoperability problems with FITS libraries in actual
use in 2022.

Writers should in general not write cards with empty values.  Readers,
however, should be prepared to see all-blank strings being used as null
values, as many files have been written assuming such a convention.

At this point, writers are encouraged to write uncompressed FITS files
for interoperability, in which case the header described here will be
the header of the primary HDU.  We do, however, envisage the use of
compressed FITS images.  In this case, the header structure discussed
here would be found in the header of the second HDU.


\subsection{Group 1 -- Basic FITS and Array Description Keywords}

This group contains the minimal keywords necessary to interpret the
HDU's data section as a pixel array.

\begin{inlinetable}
\footnotesize
\begin{tabular}{llp{0.6\textwidth}}
\sptablerule
\textbf{Keyword}&\textbf{Type}&\textbf{Description}\\
\sptablerule
SIMPLE     &logical     &(FITS Standard) File conforms to FITS Standard.\\
BITPIX     &integer     &(FITS Standard) Number of bits per data pixel.\\
NAXIS      &integer     &(FITS Standard) Number of data axes.\\
NAXIS1     &integer     &(FITS Standard) Length of data axis 1
(number of pixels in a row).\\
NAXIS2     &integer     &(FITS Standard) Length of data axis 2
(number of rows).\\
BSCALE     &float             &(FITS Standard) Scaling factor to apply
to values in the pixel array\\
BZERO      &integer     &(FITS Standard) Offset to add to values in the
pixel array\\
\end{tabular}
\end{inlinetable}

For unsigned 16-bit integer data, set BSCALE=1.0 and BZERO=32768.

\begin{fitsexample}{Array description keywords}
\begin{lstlisting}
SIMPLE  =                    T / file conforms to FITS standard
BITPIX  =                   16 / number of bits per data pixel
NAXIS   =                    2 / number of data axes
NAXIS1  =                18904 / length of data axis 1
NAXIS2  =                18904 / length of data axis 2
BSCALE  =                  1.0 / physical_value = BZERO + BSCALE * array_value
BZERO   =                32768 / physical_value = BZERO + BSCALE * array_value
\end{lstlisting}
\end{fitsexample}

\subsection{Group 2 -- Original Data of the Observation}

This group captures the original information about
the observation, as obtained from observation logbooks or similar
sources.

Most of the values in this section are not intended to be (fully)
machine-readable.  This is partly because we could not hope to give a
conclusive and useful vocabulary of, say, filters used over the 100 years
astronomical plates were taken in.  It is also because most of
this information will only be necessary when following an individual
plate's provenance; in that case, the information will be consumed by a
human anyway.

Even so, data providers are urged to be at least internally consistent
with observer names or the designations of filters, instruments, or time
systems.  Nonetheless, several pieces of metadata important for automatic
processing in many interesting use cases are present here, too, for
instance, \cardname{SITELONG}, \cardname{SITELAT}, and
\cardname{SITELEV}.  Their operational importance is also
the reason why they are given in decimal degrees (rather than, say, a
sexagesimal character string that you will probably find in the
historical documents).


\begingroup
\footnotesize
\begin{longtable}{llp{0.6\textwidth}}
\sptablerule
\textbf{Keyword}&
\textbf{Type}&\textbf{Description}\\
\sptablerule
\endhead
DATEORIG  &string  &
  Original recorded date of the observation (evening date)\\
TMS-ORIG  &string  &
  Original recorded time of the start of the observation\\
TME-ORIG  &string  &
  Original recorded time of the end of the observation.\\
TIMEFLAG  &string  &
  Quality flag of the recorded observation time: 'error', 'missing',
  'uncertain'.  Do not give for good times.\\
RA-ORIG   &string  &
  Original recorded right ascension of the telescope pointing (plate center)\\
DEC-ORIG  &string  &
  Original recorded declination of the telescope pointing (plate center)\\
EQ-ORIG   &string  &
  Equinox of equatorial coordinates in RA-ORIG and DEC-ORIG\\
COORFLAG  &string  &
  Quality flag of the recorded coordinates (right ascension and
  declination): 'error', 'missing', 'uncertain'.  Do not give for good
  positions.\\
OBJECT    &string &
  (FITS Standard) Name of the target object or field of the
  observations.  When the observation had multiple targets, put
  'multiple' here and give
  individual target names in OBJECTn cards.\\
OBJTYPE   &string  &
  Target object type (literal text); controlled vocabulary, see below\\
EXPTIME   &float   &
  Exposure time of the first exposure, expressed in seconds\\
NUMEXP    &integer &Number of exposures\\
DATEOR$n$ &string  &
  Original recorded date of the n-th exposure if exposures
  were made on multiple nights. Not used when all exposures are from
  one night, given by DATEORIG.\\
TMS-OR$n$ &string  &
  Original recorded time of the start of the n-th exposure.\\
TME-OR$n$ &string  &
  Original recorded time of the end of the n-th exposure.\\
RA-OR$n$  &string  &
  Original right ascension of the telescope pointing during the n-th
  exposure. Not used if only one pointing was used.\\
DEC-OR$n$ &string   &
  Original declination of the telescope pointing during the n-th
  exposure. Not used if only one pointing was used.\\
OBJECT$n$ &string  &
    Object (field) name on the n-th exposure. Not used if only
    one object (field) was observed.\\
OBJTYP$n$ &string  &Object type that corresponds to
OBJECT$n$; see OBJTYPE\\
EXPTIM$n$ &float   & Exposure time of the n-th exposure\\
OBSERVAT  &string  &Observatory name\\
SITENAME  &string  &
  Observatory site name. Useful if the observatory has more than one
  observing site.\\
SITELONG  &float   &
  East longitude of the observing site, in decimal degrees\\
SITELAT   &float   &
  Latitude of the observing site, in decimal degrees\\
SITEELEV  &float   &
  Elevation of the observatory site [m]\\
TELESCOP  &string  &(FITS Standard) Telescope name\\
OTA-NAME  &string  &Free-text designation of the optical tube assembly (OTA)\\
OTA-DIAM  &float   &Diameter of the OTA (e.g., primary mirror; different
from OTA-APER in instruments like Schmidt telescopes [m]\\
OTA-APER  &float   &Clear aperture of the OTA [m]\\
FOCLEN    &float   &Focal length of the telescope [m]\\
PLTSCALE  &float   &Plate scale of the telescope [arcsec/mm]\\
INSTRUME  &string  &(FITS Standard) Instrument name\\
DETNAM    &string  &Detector name; rather typcially, 'photographic plate'\\
METHOD    &string  &
  Observation method.  See below for values allowed.\\
FILTER    &string  &Free text designation of a filter (if used)\\
PRISM     &string  &
  Free text designation on the objective prism (if used)\\
PRISMANG  &string  &
  Angle of the objective prism (format deg:min)\\
DISPERS   &float   &Dispersion in objective prism images [Angstrom/mm]\\
GRATING   &string  &Free text on the grating (if used)\\
FOCUS     &float   & Focus value (from logbook).\\
FOCUS$n$  &float   &
  Focus value of the n-th exposure; used instead of FOCUS when several
  exposures with different focus values were made.\\
TEMPERAT  &float   &Air temperature from logbook.\\
CALMNESS  &string  &
  Calmness (seeing conditions), scale 1–5\\
SHARPNES  &string  &Sharpness (seeing conditions), scale 1–5\\
TRANSPAR  &string  &
  Transparency, scale 1–5\\
SEEING    &string  &Quantitative statement on the seeing, typically as
  an angle (include the unit in that case)\\
SKYCOND   &string  &Notes on sky conditions from logbook\\
OBSERVER  &string  &(FITS Standard) Observer name\\
OBSNOTES  &string  &Observer notes from logbook\\
NOTES     &string  &Miscellaneous notes found in the observation
logbook.\\
\end{longtable}
\endgroup

The notion of the ``optical tube assembly'' (OTA) is used here because
in several cases multiple tubes shared a single mounting; calling each of
these tubes a separate ``telescope'' is not in line with what the word
was used for historically.  We hence decided to use a new, dedicated
term here.  Other examples for where this notion helps include
diaphragms used to stop down the aperture or instruments with multiple
secondaries.

The values in \cardname{TMS-ORIG}, \cardname{TME-ORIG},
\cardname{TMS-ORn}, and \cardname{TMS-ORn} should have the format
``TZ hh:mm:ss'', where TZ denotes the time system. These values should not be
transformed from what was taken from the logbook, except when
correcting obvious (spelling) mistakes.  Historically, all kinds of
times were used on observatories, and hence we do not attempt to make
time system identifiers properly machine-readable.  Still, for
local sidereal time, \verb|ST| should be used, for universal time or
Greenwhich Mean Time, \verb|UT|.  The use of other designations should
be explained, e.g., in accompanying documentation and/or FITS comments.
Multiple time notations are separated with commas (e.g. 'UT 18:13, ST
02:44').  Again, this data will generally be parsed by humans.

In case of multiple exposures (\cardname{NUMEXP} is greater than 1),
exposure times of all sub-exposures are to be given with the
\cardname{EXPTIMn} keywords.  The \cardname{EXPTIME} and
\cardname{EXPTIM1} cards then have the same value.  This is different in
the (relatively rare) cases of exposures with multiple focus values.
For such plates, no \cardname{FOCUS} is given but only
\cardname{FOCUS$n$}.

\cardname{EQ-ORIG} should give the equinox of the coordinates in
\cardname{RA-ORIG}, \cardname{DEC-ORIG}.  If given, it must be in
years, with a B or J for Besselian or Julian years (as in J2000.0 or
B1950.0).  A plain floating point value is legal, too, and indicates
an unnknown year convention.  \cardname{RA-OR$n$} and
\cardname{DEC-OR$n$} can be assumed to share \cardname{EQ-ORIG}.

When multiple \cardname{OBJECT$n$} and \cardname{TMS-OR$n$} cards are
present, the relationship of exposures and target objects is not defined
by this specification; it can mean one target object per exposure,
multiple exposures of the same set of target objects, or any combination
thereof.

\cardname{CALMNESS} and \cardname{SHARPNESS} would probably both map to
a modern seeing value; they were introduced to tell apart ``Ruhe'' and
``Schärfe'' that are common in German log books.

\cardname{OBJTYPE} and \cardname{OBJTYP$n$}, where present and
non-blank, has to contain one of the following values (taken from the
WFPDB):

\begin{quotation}
\noindent comet, HII region, planetary nebula, reference star around a radio
source, star, moon, field, asteroid, fundamental star
, cluster of galaxies, nebula, galaxy, star cluster, planet, double
star, sun, QSO, variable star, supernova, group of galaxies
\end{quotation}

Where present, \cardname{METHOD} has to contain one of the following
values (again taken from the WFPDB):

\begin{quotation}
\noindent test plate, direct photograph, multiexposure, stellar tracks, sub-beam
(Pickering) prism, with mask, objective prism, Hartmann test, out of
focus, objective prism, multiexposure, no guiding, objective grating,
direct photograph, Metcalf's method
\end{quotation}

If a grating has been used, additional freetext should be given in
\cardname{GRATING}; this could be text like  ``Coarse objective
grating'' (as used for bright-star photometry) or ``Spectroscopy
objective grating'' as appropriate.

\cardname{DETNAME} would typcially be the fixed string ``photographic
plate'', but other media are conceivable, in particular ``film''.
Please avoid being more specific here and rather do finer-grained
descriptions \cardname{EMULSION} or \cardname{OAT-NAME}.

For \cardname{CALMNESS}, \cardname{SHARPNES}, and \cardname{TRANSPAR}, a
numeric scale of 1-5 is used here, where 1 is best and 5 is worst.
These keywords are probably too specific and may be removed in future
versions of this document.  Use free-text in \cardname{SKYCOND} instead
or, where applicable, \cardname{SEEING}.

\begin{fitsexample}{Observation data, multiple exposures}
\begin{lstlisting}
EXPTIME =                600.0 / [s] exposure time (of exposure 1)
NUMEXP  =                    3 / number of exposures of the plate
EXPTIM1 =                600.0 / [s] exposure time of exposure 1
EXPTIM2 =                 60.0 / [s] exposure time of exposure 2
EXPTIM3 =                  2.0 / [s] exposure time of exposure 3
\end{lstlisting}
\end{fitsexample}

\begin{fitsexample}{Observation data, single exposure}
\begin{lstlisting}
EXPTIME =               1800.0 / [s] exposure time (of exposure 1)
NUMEXP  =                    1 / number of exposures of the plate
\end{lstlisting}
\end{fitsexample}


\begin{fitsexample}{Observation data, full header}
\begin{lstlisting}
        --------------------------------------- Original data of the observation
DATEORIG= '1910-08-02'         / recorded date of the observation
TMS-ORIG= 'ST 18:11:16'        / recorded time of the start of the observation
TME-ORIG= '        '           / recorded time of the end of the observation
TIMEFLAG= 'uncertain'          / quality of the recorded time
RA-ORIG = '19:11:42'           / recorded right ascension of telescope pointing
DEC-ORIG= '15:04:00'           / recorded declination of telescope pointing
COORFLAG= 'uncertain'          / quality of the recorded coordinates
OBJECT  = 'SA 87   '           / name of the observed object or field
OBJTYPE = 'field   '           / object type
EXPTIME =               1800.0 / [s] exposure time (of exposure 1)
NUMEXP  =                    1 / number of exposures of the plate
OBSERVAT= 'Astrophysikalische Observatorium Potsdam' / observatory name
SITENAME= 'Potsdam-Telegrafenberg' / observatory site name
SITELONG=            13.064167 / [deg] East longitude of the observatory
SITELAT =            52.380556 / [deg] latitude of the observatory
SITEELEV=                  107 / [m] elevation of the observatory
TELESCOP= 'Zeiss Triplet'      / telescope name
OTA-NAME= '15-cm camera'       / optical tube assembly (OTA)
OTA-DIAM=                 0.15 / [m] diameter of the OTA
OTA-APER=                 0.15 / [m] clear aperture of the OTA
FOCLEN  =                  1.5 / [m] focal length of the OTA
PLTSCALE=                  137 / [arcsec/mm] plate scale of the OTA
INSTRUME= '        '           / instrument
DETNAM  = 'photographic plate' / detector
METHOD  = 'direct photograph'  / method of observation
FILTER  = 'none    '           / filter type
PRISM   = '        '           / objective prism
PRISMANG= '        '           / prism angle "deg:min"
DISPERS =                      / [Angstrom/mm] dispersion
GRATING = '        '           / grating
FOCUS   =                 32.2 / focus value
TEMPERAT=                 21.8 / [deg C] air temperature (degrees Celsius)
CALMNESS= '2-3     '           / sky calmness (scale 1-5)
SHARPNES= '2       '           / sky sharpness (scale 1-5)
TRANSPAR= '1-2     '           / sky transparency (scale 1-5)
SKYCOND = 'moonlight'          / sky conditions
OBSERVER= 'W. Muench'          / observer name
OBSNOTES= 'bad guiding'        / observer notes
NOTES   = 'SA 87 = Kapteyn Selected Area 87' / miscellaneous notes
\end{lstlisting}
\end{fitsexample}

\begin{fitsexample}{Observation data, multiple time systems}
\begin{lstlisting}
        --------------------------------------- Original data of the observation
DATEORIG= '1964-01-02'         / recorded date of the observation
TMS-ORIG= 'UT 18:13, ST 02:44' / recorded time of the start of exposure 1
TME-ORIG= 'UT 19:13, ST 03:44' / recorded time of the end of exposure 1
TIMEFLAG= '        '           / quality of the recorded time
EXPTIME =               3600.0 / [s] exposure time (of exposure 1)
NUMEXP  =                    1 / number of exposures of the plate
\end{lstlisting}
\end{fitsexample}


\begin{fitsexample}{Full header for multiple exposures}

\begin{lstlisting}
        --------------------------------------- Original data of the observation
DATEORIG= '1934-04-01'         / recorded date of the observation
TMS-OR1 = 'ST 10:52'           / recorded time of the start of exposure 1
TMS-OR2 = 'ST 10:54'           / recorded time of the start of exposure 2
TMS-OR3 = 'ST 10:57'           / recorded time of the start of exposure 3
TME-OR1 = 'ST 10:53'           / recorded time of the end of exposure 1
TME-OR2 = 'ST 10:56'           / recorded time of the end of exposure 2
TME-OR3 = 'ST 11:01'           / recorded time of the end of exposure 3
TIMEFLAG= '        '           / quality of the recorded time
RA-ORIG = '        '           / recorded right ascension of telescope pointing
DEC-ORIG= '        '           / recorded declination of telescope pointing
COORFLAG= 'missing '           / quality of the recorded coordinates
OBJECT  = 'RY UMa  '           / name of the observed object or field
OBJTYPE = 'variable star'      / object type
EXPTIME =                 60.0 / [s] exposure time (of exposure 1)
NUMEXP  =                    3 / number of exposures of the plate
EXPTIM1 =                 60.0 / [s] exposure time of exposure 1
EXPTIM2 =                120.0 / [s] exposure time of exposure 2
EXPTIM3 =                240.0 / [s] exposure time of exposure 3
OBSERVAT= 'Astrophysikalische Observatorium Potsdam' / observatory name
SITENAME= 'Potsdam-Telegrafenberg' / observatory site name
SITELONG=            13.064167 / [deg] East longitude of the observatory
SITELAT =            52.380556 / [deg] latitude of the observatory
SITEELEV=                  107 / [m] elevation of the observatory
TELESCOP= 'Zeiss Triplet'      / telescope name
OTA-NAME= '15-cm camera'       / optical tube assembly (OTA)
OTA-DIAM=                 0.15 / [m] diameter of the OTA
OTA-APER=                 0.15 / [m] clear aperture of the OTA
FOCLEN  =                  1.5 / [m] focal length of the OTA
PLTSCALE=                  137 / [arcsec/mm] plate scale of the OTA
INSTRUME= '        '           / instrument
DETNAM  = 'photographic plate' / detector
METHOD  = 'direct photograph, multi-exposure' / method of observation
FILTER  = 'none    '           / filter type
PRISM   = '        '           / objective prism
PRISMANG= '        '           / prism angle "deg:min"
DISPERS =                      / [Angstrom/mm] dispersion
GRATING = '        '           / grating
FOCUS   =                 34.4 / focus value
TEMPERAT=                    8 / [deg C] air temperature (degrees Celsius)
CALMNESS= '        '           / sky calmness (scale 1-5)
SHARPNES= '        '           / sky sharpness (scale 1-5)
TRANSPAR= '        '           / sky transparency (scale 1-5)
SKYCOND = 'clouds  '           / sky conditions
OBSERVER= 'W. Muench'          / observer name
OBSNOTES= 'poor transparency'  / observer notes
NOTES   = '        '           / miscellaneous notes
\end{lstlisting}
\end{fitsexample}

\subsection{Group 3 – Information about the Photographic Plate}

This group contains information about the physical plate; this includes
its size, the emulsion used, and similar pieces of metadata.  This group
is also where administrative information on the plate management is
being kept.

\begin{inlinetable}
\footnotesize
\begin{tabular}{llp{0.5\textwidth}}
\sptablerule
\textbf{Keyword}&\textbf{Type}&\textbf{Description}\\
\sptablerule
PLATENUM &string &Plate number in original observation
catalogue\\
WFPDB-ID &string &Plate identification in the WFPDB\\
SERIES   &string &Series or survey to which the plate belongs,
e.g., Carte du Ciel, Kapteyn Selected Areas\\
PLATESZ1 &float  &Plate size along axis 1 [cm]\\
PLATESZ2 &float  &Plate size along axis 2 [cm]\\
EMULSION &string &Type of the photographic emulsion\\
DEVELOP  &string &Plate development information (developer,
time)\\
PQUALITY &string &Free text on the quality of the plate\\
PLATNOTE &string &Other notes about the plate (free text; e.g., on
availability or historical relevance)\\
PRE-PROC  &string &Free text on processing of the plate before
scanning\\
\end{tabular}
\end{inlinetable}

Again, most of this information targets humans rather than machines.  In
particular, computers are not expected to be able to interpret
\cardname{PLATENUM}, \cardname{EMULSION}, or \cardname{PQUALITY}.
Individual archives should try to be internally consistent in the
contents of these fields.

\cardname{EMULSION} may also contain information on whether plates
were baked or treated to enhance sensitivity.

\cardname{PLATESZ1} and \cardname{PLATESZ2} represent the dimensions of
the glass carrier (or other medium), not of the sensitive area or of the area scanned.
One therefore cannot repliably compute the pixel density of the scan by
dividing \cardname{NAXIS1} by \cardname{PLATESZ1} (although this will in
general yield the right order of magnitude).

Several scanning projects remove markings on the plates before
scanning.  These should write ``plate markings removed'' in a
\cardname{PRE-PROC} header.

\begin{fitsexample}{Plate metadata}
\begin{lstlisting}
        ----------------------------------------------------- Photographic plate
PLATENUM= '317     '           / plate number in original observation catalogue
WFPDB-ID= 'POT015_000317'      / plate identification in the WFPDB
SERIES  = 'Kapteyn Selected Areas' / plate series
PLATEFMT= '20x20   '           / plate format in cm
PLATESZ1=                 20.0 / [cm] plate size along axis 1
PLATESZ2=                 20.0 / [cm] plate size along axis 2
EMULSION= 'Schleussner'        / photographic emulsion type
DEVELOP = '        '           / plate development information
PQUALITY= 'broken  '           / quality of plate
PLATNOTE= 'contact copy of original plate that is not available' / plate notes
\end{lstlisting}
\end{fitsexample}

\subsection{Group 4 – Computed Data For the Observation}

This group by and large contains data derived from what is given in
group 2, only using more modern units, formats, and frames.  Machines
ought to use the information given here to locate the plate in time and,
where WCS is missing, space.


\begingroup
\footnotesize
\begin{longtable}{llp{0.5\textwidth}}
\sptablerule
\textbf{Keyword}&\textbf{Type}&\textbf{Description}\\
\sptablerule
DATE-OBS  & string     &(FITS Standard) UT date and time of the
          start of the observation (format YYYY-MM-DDThh:mm:ss, or
          YYYY-MM-DD if time is not specified). The date may differ
          from DATEORIG, because the original date usually refers to the
          evening of the observing night.\\
DT-OBS$n$ &string     &UT date and time of the start of the
$n$-th exposure.\\
DATE-AVG  &string     &(FITS Standard) UT date and time of the
mid-point of the first exposure (format YYYY-MM-DDThh:mm:ss)\\
DT-AVG$n$ &string     &UT date and time of the mid-point of
the $n$-th exposure.\\
DATE-END  &string     &UT date and time of the end of the
          first exposure (format YYYY-MM-DDThh:mm:ss)\\
DT-END$n$ &string     &UT date and time of the end of the $n$-th
exposure.\\
YEAR      &float     &Decimal year of the start of the first
exposure\\
YEAR$n$   &float     &Decimal year of the start of the $n$-th
exposure\\
YEAR-AVG  &float     &Decimal year of the mid-point of the
first exposure\\
YR-AVG$n$ &float     &Decimal year of the mid-point of the
$n$-th exposure\\
JD        &float     &Julian date at the start of exposure 1\\
JD$n$     &float     &Julian date at the start of the $n$-th exposure\\
JD-AVG    &float     &Julian date at the mid-point of the
first exposure\\
JD-AVG$n$ &float     &Julian date at the mid-point of the $n$-th
exposure.\\
HJD-AVG   &float     &Heliocentric Julian date at the
mid-point of the first exposure\\
HJD-AV$n$ &float     &Heliocentric Julian date at the
mid-point of the $n$-th exposure.\\
RA        &string    & Right ascension of the telescope
pointing (ICRS, sexagesimal format h:m:s)\\
DEC       &string    & Declination of the telescope pointing
(ICRS, sexagesimal format d:m:s)\\
RA$n$     &string    & Right ascension of the telescope
pointing, $n$-th exposure\\
DEC$n$    &string    & Declination of the telescope pointing,
$n$-th exposure.\\
RA\_DEG   &float    & Right ascension of the telescope
pointing in decimal degrees (ICRS)\\
DEC\_DEG  &float    & Declination of the telescope pointing in
decimal degrees (ICRS)\\
RA\_DEG$n$&float    & Right ascension of the telescope
pointing in decimal degrees, $n$-th exposure.\\
DEC\_DE$n$&float    & Declination of the telescope pointing in
decimal degrees, $n$-th exposure.\\
\end{longtable}
\endgroup

For historical reasons, \cardname{RA} and \cardname{DEC} are still given
sexagesimally.  Writers should make sure they also give their decimal
equivalents (\cardname{RA\_DEG}, \cardname{DEC\_DEG}).

The \cardname{HJD-AVG} headers are to be understood as
lighttime-corrected universal time for the solar system barycenter.  For
the other times, no lighttime correction should be applied (the time
frame is TOPOCENTER in VO terms).  Where we write ``UT'' here, we mean a
reasonable approximation of our modern UT at the time, presumably GMT
over most of the time photographic plates were taken.  Times given in
Ephemeris Time (or perhaps even Terrestial Time) should be converted
(though the difference probably rarely matters for the sort of data this
specification talks about).

Fractional seconds are allowed on both ISO-like time specifications
(\cardname{DATE-$x$}, \cardname{DT-$x$} and on sexagesimal positions
(\cardname{RA}, \cardname{RA$n$}, \cardname{DEC}, \cardname{DEC$n$}).


\begin{fitsexample}{Computed metadata, single exposure}
\begin{lstlisting}
        --------------------------------------- Computed data of the observation
DATE-OBS= '1910-08-02T22:21:01' / UT date of the start of the observation
DATE-AVG= '1910-08-02T22:36:01' / UT date of the mid-point of exposure 1
DATE-END= '1910-08-02T22:51:01' / UT date of the end of exposure 1
YEAR    =       1910.583561644 / decimal year of the start of exposure 1
YEAR-AVG=       1910.583561644 / decimal year of the mid-point of exposure 1
JD      =       2418886.441678 / Julian date at the start of exposure 1
JD-AVG  =       2418886.441678 / Julian date at the mid-point of exposure 1
HJD-AVG =       2418886.441678 / heliocentric JD at the mid-point of exposure 1
RA      = '19:15:48'           / right ascension of pointing (J2000) "h:m:s"
DEC     = '+15:13:20'          / declination of pointing (J2000) "d:m:s"
RA_DEG  =           288.950000 / [deg] right ascension of pointing (J2000)
DEC_DEG =            15.222222 / [deg] declination of pointing (J2000)
\end{lstlisting}
\end{fitsexample}


\begin{fitsexample}{Computed metadata, multiple exposures}
\begin{lstlisting}
        --------------------------------------- Computed data of the observation
DATE-OBS= '1934-01-25T20:36:56' / UT date of the start of exposure 1
DT-OBS1 = '1934-01-25T20:36:56' / UT date of the start of exposure 1
DT-OBS2 = '1934-01-25T20:45:55' / UT date of the start of exposure 2
DT-OBS3 = '1934-01-25T20:55:53' / UT date of the start of exposure 3
DT-OBS4 = '1934-01-25T20:57:53' / UT date of the start of exposure 4
DATE-AVG= '1934-01-25T20:40:56' / UT date of the mid-point of exposure 1
DT-AVG1 = '1934-01-25T20:40:56' / UT date of the mid-point of exposure 1
DT-AVG2 = '1934-01-25T20:48:25' / UT date of the mid-point of exposure 2
DT-AVG3 = '1934-01-25T20:56:23' / UT date of the mid-point of exposure 3
DT-AVG4 = '1934-01-25T20:58:53' / UT date of the mid-point of exposure 4
DATE-END= '1934-01-25T20:44:55' / UT date of the end of exposure 1
DT-END1 = '1934-01-25T20:44:55' / UT date of the end of exposure 1
DT-END2 = '1934-01-25T20:50:54' / UT date of the end of exposure 2
DT-END3 = '1934-01-25T20:56:53' / UT date of the end of exposure 3
DT-END4 = '1934-01-25T20:59:52' / UT date of the end of exposure 4
YEAR    =        1934.06806018 / decimal year of the start of exposure 1
YEAR1   =        1934.06806018 / decimal year of the start of exposure 1
YEAR2   =        1934.06807726 / decimal year of the start of exposure 2
YEAR3   =        1934.06809621 / decimal year of the start of exposure 3
YEAR4   =        1934.06810001 / decimal year of the start of exposure 4
YEAR-AVG=        1934.06806779 / decimal year of the mid-point of exposure 1
YR-AVG1 =        1934.06806779 / decimal year of the mid-point of exposure 1
YR-AVG2 =        1934.06808202 / decimal year of the mid-point of exposure 2
YR-AVG3 =        1934.06809716 / decimal year of the mid-point of exposure 3
YR-AVG4 =        1934.06810192 / decimal year of the mid-point of exposure 4
JD      =        2427463.35898 / Julian date at the start of exposure 1
JD1     =        2427463.35898 / Julian date at the start of exposure 1
JD2     =        2427463.36522 / Julian date at the start of exposure 2
JD3     =        2427463.37214 / Julian date at the start of exposure 3
JD4     =        2427463.37353 / Julian date at the start of exposure 4
JD-AVG  =        2427463.36176 / Julian date at the mid-point of exposure 1
JD-AVG1 =        2427463.36176 / Julian date at the mid-point of exposure 1
JD-AVG2 =        2427463.36696 / Julian date at the mid-point of exposure 2
JD-AVG3 =        2427463.37249 / Julian date at the mid-point of exposure 3
JD-AVG4 =        2427463.37422 / Julian date at the mid-point of exposure 4
HJD-AVG =                      / heliocentric JD at the mid-point of exposure 1
\end{lstlisting}
\end{fitsexample}

Note that this example does not give a position; this is rather common
when, for instance, the exposure was guided on a solar system object.


\subsection{Group 5 – Scan Details}

This group contains information about the scanning process.  Except
possibly for \cardname{WEDGE}, we do not expect that these headers will
be automatically processed by machines.

\begin{inlinetable}
\footnotesize
\begin{tabular}{llp{0.5\textwidth}}
\sptablerule
\textbf{Keyword}&\textbf{Type}&\textbf{Description}\\
\sptablerule
SCANRES1 &integer &Scan resolution along axis 1 [dpi]\\
SCANRES2 &integer &Scan resolution along axis 2 [dpi]\\
PIXSIZE1 &float   &Pixel size along axis 1 [$\mu\rm m$]\\
PIXSIZE2 &float   &Pixel size along axis 2 [$\mu\rm m$]\\
SCANSOFT &string  &Name of the scanning software\\
SCANGAM  &float   &Scan gamma value\\
SCANFOC  &string  &Scan focus (e.g., 'glass')\\
WEDGE    &string  &Type of photometric step-wedge\\
DATESCAN &string  &
  Scan date and time (UTC, format "YYYY-MM-DDThh:mm:ss")\\
SCANAUTH &string  &Author of the scan\\
SCANNOTE &string  &
  Free text notes about the scan (e.g., scan orientation)\\
\end{tabular}
\end{inlinetable}


By the FITS Standard, the \cardname{AUTHOR} and \cardname{REFERENCE} 
keywords are used when the
data in the FITS file was compiled from a publication or multiple
sources. For digitised photographic plates, these keywords are not
appropriate for specifying the author of the scan or acknowledging a
funding source. Hence, we add the \cardname{SCANAUTH} header with
similar semantics as \cardname{AUTHOR}, just regarding the scanning
process.  Acknowledgments (as in \cardname{REFERENCE}) would currently
go into a comment in group~8.


\begin{fitsexample}{Scanning metadata}
\begin{lstlisting}
        ----------------------------------------------- Scan Details
SCANNER = 'Epson Expression 10000XL' / scanner name
SCANRES1=                 2400 / [dpi] scan resolution along axis 1
SCANRES2=                 2400 / [dpi] scan resolution along axis 2
PIXSIZE1=              10.5833 / [um] pixel size along axis 1
PIXSIZE2=              10.5833 / [um] pixel size along axis 2
SCANSOFT= 'VueScan '           / scanning software
SCANGAM =                  1.0 / scan gamma value
SCANFOC = 'glass   '           / scan focus
WEDGE   = 'Danes-Picta TG21S'  / type of photometric step-wedge
DATESCAN= '2011-05-17T08:33:26' / scan date and time
SCANAUTH= 'K. Tsvetkova'       / author of scan
SCANNOTE= 'Plate rotated 90 degrees' / scan notes
\end{lstlisting}
\end{fitsexample}


\subsection{Group 6 – Data Files}

This group mainly contains references to digital artefacts resulting
from the scanning process.  The values of these can either be plain file
names or, preferably, full URIs.  Where plain file names are given, a
FITS comment should give further information on where these files might
be found.

\begin{inlinetable}
\footnotesize
\begin{tabular}{llp{0.5\textwidth}}
\sptablerule
\textbf{Keyword}&\textbf{Type}&\textbf{Description}\\
\sptablerule
FILENAME &string    &Filename of the plate scan (this file)\\
FN-SCNn  &string    &
  Filename of the $n$-th scan of the same plate\\
FN-WEDGE &string    &Filename of the wedge scan\\
FN-PRE   &string    &
  Filename of a low-resolution scan\\
FN-COVER &string    &
  Filename of the plate cover (envelope) image\\
FN-LOGB  &string    &Filename of the logbook image\\
FN-NTBn  &string    &
  Filename of the $n$-th notebook image\\
ORIGIN   &string    &
  (FITS Standard) Institute responsible for creating the FITS file\\
DATE     &string    &
  (FITS Standard) Date and time of the last change of the file \\
\end{tabular}
\end{inlinetable}

\cardname{FN-PRE} does not mean a thumbnail, but a compact, lossy
representation -- typically, a JPEG image.  Within the APPLAUSE project,
this low-resolution image was used to document annotations on the plates
before they were erased in preparation of the full-resolution scan.


\begin{fitsexample}{Associated files}
\begin{lstlisting}
        ------------------------------------------------------------- Data files
FILENAME= 'POT015_000317.fits' / filename of the plate scan
FN-SCN1 = 'POT015_000317.fits' / filename of scan 1
FN-WEDGE= 'POT015_000317_w.fits' / filename of the wedge scan
FN-PRE  = 'POT015_000317_pre.jpg' / filename of the preview image
FN-COVER= '        '           / filename of the plate cover image
FN-LOG1 = 'ZT-LB01-000317-000334.jpg' / filename of logbook image 1
ORIGIN  = 'Leibniz-Institut fuer Astrophysik Potsdam (AIP)'
DATE    = '2022-03-01T07:55:13' / last change of this file
\end{lstlisting}
\end{fitsexample}

As stated above, giving a blank values -- as in \cardname{FN-COVER} in
this example -- is the less preferred alternative to leaving out a card
without a value.


\begin{fitsexample}{Associated files, multiple notebook pages}
\begin{lstlisting}
        ------------------------------------------------------------- Data files
FILENAME= 'LA00508_x.fits'     / filename of the plate scan
FN-SCN1 = 'LA00508_y.fits'     / filename of scan 1
FN-SCN2 = 'LA00508_x.fits'     / filename of scan 2
FN-WEDGE= '        '           / filename of the wedge scan
FN-PRE  = 'LA00508_pre.jpg'    / filename of the preview image
FN-COVER= 'LA00508_cover.jpg'  / filename of the plate cover image
FN-LOG1 = 'LA-PV01-LA00501_00510.jpg' / filename of logbook image 1
FN-LOG2 = 'LA-LB04-1916-10-18a.jpg' / filename of logbook image 2
FN-LOG3 = 'LA-LB04-1916-10-18b.jpg' / filename of logbook image 3
FN-LOG4 = 'LA-LB04-1916-10-18c.jpg' / filename of logbook image 4
FN-LOG5 = 'LA-LB04-1916-10-18d.jpg' / filename of logbook image 5
FN-LOG6 = 'LA-LB04-1916-10-18e.jpg' / filename of logbook image 6
FN-LOG7 = 'LA-LB04-1916-10-18f.jpg' / filename of logbook image 7
ORIGIN  = 'Hamburger Sternwarte (Universitaet Hamburg)'
DATE    = '2021-12-14T17:43:44' / last change of this file
\end{lstlisting}
\end{fitsexample}

\subsection{Group 7 – World Coordinate System (WCS)}

Whenever possible, plate scans should come with a WCS solution as per
\citet{2002A&A...395.1077C}.  In particular for wide-field plates, we
recommend that polynomial (SIP) corrections be included.  Approximate
WCS solutions based on the observation log are permitted but
discouraged.

\begin{fitsexample}{WCS}
\begin{lstlisting}
        -------------- World Coordinate System (WCS)
WCSAXES = 2 / number of axes in the WCS description
RADESYS = 'FK5'  / name of the reference frame
EQUINOX = 2000.0 / epoch of the mean equator and equinox in years
CTYPE1 = 'RA-TAN' / TAN (gnomonic) projection
CTYPE2 = 'DEC-TAN' / TAN (gnomonic) projection
CUNIT1 = 'deg' / physical units of CRVAL and CDELT for axis 1
CUNIT2 = 'deg' / physical units of CRVAL and CDELT for axis 2
CRPIX1 = 9452.5 / reference pixel for axis 1
CRPIX2 = 9452.5 / reference pixel for axis 2
CRVAL1 = 288.95 / right ascension at the reference point
CRVAL2 = 15.222222 / declination at the reference point
CD1_1 = -0.0004047524 / transformation matrix
CD1_2 = 0.0 / transformation matrix
CD2_1 = 0.0 / transformation matrix
CD2_2 = 0.0004047524 / transformation matrix
LONPOLE = 0.0 / native longitude of the celestial pole
\end{lstlisting}
\end{fitsexample}


\subsection{Group 8 – modification history and acknowledgements}

The eighth group collects global free-text history and global comments; of
course, comments on individual items can be present in other sections,
too, and processing software should keep those comment cards in their
original places (relative to the header card that follows them) when
re-writing a header.

In particular, use comment cards for acknowledgements.  None of this is
intended to be consumable by machines, except possibly the LICENCE URI.

\begin{inlinetable}
\footnotesize
\begin{tabular}{llp{0.5\textwidth}}
\sptablerule
\textbf{Keyword}&\textbf{Type}&\textbf{Description}\\
\sptablerule
LICENCE &string    &A URI pointing to conditions of use.  Prefer a
URI widely known if possible, and if you use a licence (rather than, say
CC0), specify the copyright holder in free text.\\
\end{tabular}
\end{inlinetable}


\begin{fitsexample}{Header Metadata}
\begin{lstlisting}
        ---------------------------------------------------------------- Licence
LICENCE = 'https://creativecommons.org/publicdomain/zero/1.0/'
        ------------------------------------------------------- Acknowledgements
COMMENT The digitized image is provided by Leibniz Institute for Astrophysics
COMMENT Potsdam (AIP). Funding for APPLAUSE has been provided by DFG (German
COMMENT Research Foundation), Leibniz Institute for Astrophysics Potsdam (AIP),
COMMENT Dr. Karl Remeis-Sternwarte, Bamberg (Friedrich-Alexander-Universitaet
COMMENT Erlangen-Nuernberg), Hamburger Sternwarte (Unversity Hamburg) and Tartu
COMMENT Observatory (University of Tartu). We thank Thueringer Landessternwarte
COMMENT Tautenburg (TLS), Astrophysikalisches Institut und
COMMENT Universitaetsternwarte Jena (University of Jena) and the Vatican
COMMENT Observatory for providing digitized Material for inclusion into the
COMMENT archive.
        ---------------------------------------------------------------- History
HISTORY Header updated with PyPlate v4.0.12 at 2022-03-01T07:53:29
HISTORY WCS added with PyPlate v4.0.12 at 2022-03-01T07:55:13
\end{lstlisting}
\end{fitsexample}


\section{Complete sample header}
\label{sect:samplehdr}

The following example contains many cards without a value, which we are
keeping here for completeness.  Headers on released data should not
include cards with empty values (see sect.~\ref{sect:format} on such
cards).

\begin{lstlisting}
SIMPLE  =                    T / file conforms to FITS standard
BITPIX  =                   16 / number of bits per data pixel
NAXIS   =                    2 / number of data axes
NAXIS1  =                18904 / length of data axis 1
NAXIS2  =                18904 / length of data axis 2
EXTEND  =                    T / file may contain extensions
BSCALE  =                  1.0 / physical_value = BZERO + BSCALE * array_value
BZERO   =                32768 / physical_value = BZERO + BSCALE * array_value
MINVAL  =                147.0 / minimum image value
MAXVAL  =              65532.0 / maximum image value
        --------------------------------------- Original data of the observation
DATEORIG= '1910-08-02'         / recorded date of the observation
TMS-ORIG= 'ST 19:52:08'        / recorded time of the start of exposure 1
TME-ORIG= '        '           / recorded time of the end of exposure 1
JDA-ORIG=                      / recorded Julian date, mid-point of exposure 1
TIMEFLAG= '        '           / quality flag of recorded time
RA-ORIG = '19:11:42'           / recorded right ascension of exposure 1
DEC-ORIG= '+15:04  '           / recorded declination of exposure 1
COORFLAG= '        '           / quality flag of recorded coordinates
OBJECT  = 'SA 87   '           / observed object or field (exposure 1)
OBJTYPE = 'field   '           / object type
EXPTIME =               1800.0 / [s] exposure time of exposure 1
NUMEXP  =                    1 / number of exposures of the plate
OBSERVAT= 'Astrophysikalische Observatorium Potsdam' / observatory name
SITENAME= 'Potsdam-Telegrafenberg' / observatory site
SITELONG=             13.06417 / [deg] East longitude of the observatory
SITELAT =             52.38056 / [deg] latitude of the observatory
SITEELEV=                  107 / [m] elevation of the observatory
TELESCOP= 'Zeiss Triplet'      / telescope name
OTA-NAME= '15-cm camera'       / optical tube assembly (OTA)
OTA-DIAM=                 0.15 / [m] diameter of the OTA
OTA-APER=                 0.15 / [m] clear aperture of the OTA
FOCLEN  =                  1.5 / [m] focal length of the OTA
PLTSCALE=                  137 / [arcsec/mm] plate scale of the OTA
INSTRUME= '        '           / instrument
DETNAM  = 'photographic plate' / detector
METHOD  = 'direct photograph'  / method of observation
FILTER  = 'none    '           / filter type
PRISM   = '        '           / objective prism
PRISMANG= '        '           / prism angle "deg:min"
DISPERS =                      / [Angstrom/mm] dispersion
GRATING = '        '           / grating
FOCUS   =                 32.2 / focus value
TEMPERAT=                 21.8 / [deg C] air temperature (degrees Celsius)
CALMNESS= '2-3     '           / sky calmness (scale 1-5)
SHARPNES= '2       '           / sky sharpness (scale 1-5)
TRANSPAR= '1-2     '           / sky transparency (scale 1-5)
SKYCOND = 'moonlight'          / sky conditions
OBSERVER= 'W. Muench'          / observer name
OBSNOTES= 'bad guiding'        / observer notes
NOTES   = 'SA 87 = Kapteyn Selected Area 87' / miscellaneous notes
BIBCODE = '        '           / bibcode of a related paper
        ----------------------------------------------------- Photographic plate
PLATENUM= '317     '           / plate number in archive
PNUMORIG= '        '           / original plate number in archive
WFPDB-ID= 'POT015_000317'      / plate identification in the WFPDB
SERIES  = 'Kapteyn series'     / plate series
PLATEFMT= '20x20   '           / plate format in cm
PLATESZ1=                   20 / [cm] plate size along axis 1
PLATESZ2=                   20 / [cm] plate size along axis 2
EMULSION= '        '           / photographic emulsion type
DEVELOP = '        '           / plate development details
PQUALITY= '        '           / quality of plate
PLATNOTE= '        '           / plate notes
        --------------------------------------- Computed data of the observation
DATE-OBS= '1910-08-02T22:17:23' / UT date of the start of exposure 1
DATE-AVG= '1910-08-02T22:32:23' / UT date of the mid-point of exposure 1
DATE-END=                      / UT date of the end of exposure 1
YEAR    =        1910.58570496 / decimal year of the start of exposure 1
YEAR-AVG=        1910.58573348 / decimal year of the mid-point of exposure 1
YEAR-END=                      / decimal year of the end of exposure 1
JD      =        2418886.42874 / Julian date at the start of exposure 1
JD-AVG  =        2418886.43916 / Julian date at the mid-point of exposure 1
JD-END  =                      / Julian date at the end of exposure 1
HJD-AVG =        2418886.44361 / heliocentric JD at the mid-point of exposure 1
RA      = '19:15:44.1'         / right ascension of pointing (J2000) "h:m:s"
DEC     = '15:13:29.9'         / declination of pointing (J2000) "d:m:s"
RA_DEG  =             288.9337 / [deg] right ascension of pointing (J2000)
DEC_DEG =               15.225 / [deg] declination of pointing (J2000)
        ------------------------------------------------------------------- Scan
SCANNER = 'Epson Expression 10000XL' / scanner name
SCANRES1=                 2400 / [dpi] scan resolution along axis 1
SCANRES2=                 2400 / [dpi] scan resolution along axis 2
PIXSIZE1=              10.5833 / [um] pixel size along axis 1
PIXSIZE2=              10.5833 / [um] pixel size along axis 2
SCANSOFT= 'VueScan '           / scanning software
SCANGAM =                  1.0 / scan gamma value
SCANFOC = 'glass   '           / scan focus
WEDGE   = 'Danes-Picta TG21S'  / type of photometric step-wedge
DATESCAN= '2011-05-17T08:33:26' / scan date and time
SCANAUTH= 'K. Tsvetkova'       / author of scan
SCANNOTE= 'Plate rotated 90 degrees' / scan notes
        ------------------------------------------------------------- Data files
FILENAME= 'POT015_000317.fits' / filename of the plate scan
FN-SCN1 = 'POT015_000317.fits' / filename of scan 1
FN-WEDGE= 'POT015_000317_w.fits' / filename of the wedge scan
FN-PRE  = 'POT015_000317_pre.jpg' / filename of the preview image
FN-COVER= '        '           / filename of the plate cover image
FN-LOG1 = 'ZT-LB01-000317-000334.jpg' / filename of logbook image 1
ORIGIN  = 'Leibniz-Institut fuer Astrophysik Potsdam (AIP)'
DATE    = '2022-03-01T07:55:13' / last change of this file
        -------------------------------------------------------------------- WCS
HISTORY   Astrometric solution by SCAMP version 2.10.0 (2021-06-07)
EQUINOX =        2000.00000000 / Mean equinox
RADESYS = 'ICRS    '           / Astrometric system
CTYPE1  = 'RA---TPV'           / WCS projection type for this axis
CTYPE2  = 'DEC--TPV'           / WCS projection type for this axis
CUNIT1  = 'deg     '           / Axis unit
CUNIT2  = 'deg     '           / Axis unit
CRVAL1  =   2.889496876869E+02 / World coordinate on this axis
CRVAL2  =   1.499181176825E+01 / World coordinate on this axis
CRPIX1  =   9.452500000000E+03 / Reference pixel on this axis
CRPIX2  =   9.452500000000E+03 / Reference pixel on this axis
CD1_1   =  -4.036655485153E-04 / Linear projection matrix
CD1_2   =  -3.621686854687E-06 / Linear projection matrix
CD2_1   =  -3.730596479942E-06 / Linear projection matrix
CD2_2   =   4.052335966967E-04 / Linear projection matrix
PV1_0   =   1.819365521161E-04 / Projection distortion parameter
PV1_1   =   1.001005528320E+00 / Projection distortion parameter
PV1_2   =   5.017392310243E-04 / Projection distortion parameter
PV1_4   =  -2.323791286135E-05 / Projection distortion parameter
PV1_5   =  -1.359281566747E-05 / Projection distortion parameter
PV1_6   =   1.638989950374E-05 / Projection distortion parameter
PV1_7   =  -1.356092599012E-04 / Projection distortion parameter
PV1_8   =  -5.900609933445E-06 / Projection distortion parameter
PV1_9   =  -9.583368487198E-05 / Projection distortion parameter
PV1_10  =   5.642839611951E-06 / Projection distortion parameter
PV2_0   =   6.375509204261E-04 / Projection distortion parameter
PV2_1   =   1.002544704064E+00 / Projection distortion parameter
PV2_2   =  -2.878680758084E-05 / Projection distortion parameter
PV2_4   =   6.122481161384E-04 / Projection distortion parameter
PV2_5   =  -5.500999776357E-06 / Projection distortion parameter
PV2_6   =   2.327669386965E-05 / Projection distortion parameter
PV2_7   =   1.570030554563E-04 / Projection distortion parameter
PV2_8   =   4.476899592895E-06 / Projection distortion parameter
PV2_9   =  -9.087182263726E-05 / Projection distortion parameter
PV2_10  =  -2.680250757454E-06 / Projection distortion parameter
FGROUPNO=                    1 / SCAMP field group label
ASTIRMS1=   0.000000000000E+00 / Astrom. dispersion RMS (intern., high S/N)
ASTIRMS2=   0.000000000000E+00 / Astrom. dispersion RMS (intern., high S/N)
ASTRRMS1=   1.139308494906E-04 / Astrom. dispersion RMS (ref., high S/N)
ASTRRMS2=   1.078386077369E-04 / Astrom. dispersion RMS (ref., high S/N)
ASTINST =                    1 / SCAMP astrometric instrument label
FLXSCALE=   0.000000000000E+00 / SCAMP relative flux scale
MAGZEROP=           0.00000000 / SCAMP zero-point
PHOTIRMS=           0.00000000 / mag dispersion RMS (internal, high S/N)
PHOTINST=                    1 / SCAMP photometric instrument label
PHOTLINK=                    F / True if linked to a photometric field
        ---------------------------------------------------------------- Licence
LICENCE = 'https://creativecommons.org/publicdomain/zero/1.0/'
        ------------------------------------------------------- Acknowledgements
COMMENT The digitized image is provided by Leibniz Institute for Astrophysics
COMMENT Potsdam (AIP). Funding for APPLAUSE has been provided by DFG (German
COMMENT Research Foundation), Leibniz Institute for Astrophysics Potsdam (AIP),
COMMENT Dr. Karl Remeis-Sternwarte, Bamberg (Friedrich-Alexander-Universitaet
COMMENT Erlangen-Nuernberg), Hamburger Sternwarte (Unversity Hamburg) and Tartu
COMMENT Observatory (University of Tartu). We thank Thueringer Landessternwarte
COMMENT Tautenburg (TLS), Astrophysikalisches Institut und
COMMENT Universitaetsternwarte Jena (University of Jena) and the Vatican
COMMENT Observatory for providing digitized Material for inclusion into the
COMMENT archive.
        ---------------------------------------------------------------- History
HISTORY Header updated with PyPlate v4.0.12 at 2022-03-01T07:53:29
HISTORY WCS added with PyPlate v4.0.12 at 2022-03-01T07:55:13
        -------------------------------------------------------------- Checksums
CHECKSUM= 'EUWaGSUXESUaESUU'   / HDU checksum updated 2022-03-01T08:55:18
DATASUM = '2591757796'         / data unit checksum updated 2022-03-01T08:55:18
        ------------------------------------------------------------------------
\end{lstlisting}


\appendix
\section{Changes from Previous Versions}

\subsection{Changes from APPLAUSE's Header List}

The following changes versus the header list at 
\url{https://www.plate-archive.org/applause/wiki/fits-header-format-dr2/}
were made during an informal review phase leading
up to version 1.0 of the IVOA Note.

\begin{itemize}
\item New header \cardname{EQ-ORIG} for giving the equinox of the
coordinates.
\item \cardname{DETNAM} is now explicitly allowed to be something other
than ``photographic plate''.
\item New header \cardname{SEEING}, sort-of deprecating
\cardname{CALMNESS}, \cardname{SHARPNES}, and \cardname{TRANSPAR}.
\item New header \cardname{PRE-PROC} for indicating scrubbing of plates
or the like.
\end{itemize}

\bibliography{ivoatex/ivoabib,ivoatex/docrepo,local}


\end{document}
