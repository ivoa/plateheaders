\documentclass[11pt]{ivoa}
% We need a bit of extra width for 80 tt characters/line
\usepackage[textwidth=14cm,a4paper]{geometry}
\usepackage{longtable}
\input tthdefs
\lstset{basicstyle=\ttfamily\footnotesize,flexiblecolumns=True}
\usepackage{todonotes}

\title{FITS Headers for Scans of Glass Plates}

\ivoagroup{Data Model}

\author{Tuvikene, S.}
\author{Demleitner, M.}
\author{Schmalz, S.}
\author{Enke, H.}
\author{Partl, A.}

\editor{Demleitner, M.}

% origin: https://www.plate-archive.org/applause/wiki/fits-header-format-dr2/
% \previousversion[????URL????]{????Concise Document Label????}
\previousversion{This is the first public release}
       

\begin{document}
\begin{abstract}
???? Abstract ????
\end{abstract}


\section*{Conformance-related definitions}

The words ``MUST'', ``SHALL'', ``SHOULD'', ``MAY'', ``RECOMMENDED'', and
``OPTIONAL'' (in upper or lower case) used in this document are to be
interpreted as described in IETF standard RFC2119 \citep{std:RFC2119}.

The \emph{Virtual Observatory (VO)} is a
general term for a collection of federated resources that can be used
to conduct astronomical research, education, and outreach.
The \href{https://www.ivoa.net}{International
Virtual Observatory Alliance (IVOA)} is a global
collaboration of separately funded projects to develop standards and
infrastructure that enable VO applications.


\section{Introduction}

This document  proposes FITS header keywords, value types and overall
layout for storing metadata of digitized astronomical photographic
plates. The header format is intended to suit various cases: direct
images with single exposures, multiple exposures of a single
object/field, exposures of different objects/fields, objective-prism
spectra, etc.

A FITS header format for astronomical photographic plates has been
proposed by \citet{2012PASRB..11..147K},\todo{That's probably not the
paper: Can we get it into ADS? Or can we emancipate a bit from that?}
hereafter Paper I. The introduced FITS keywords have been implemented in
the header2011 software \citep{soft:header2011}\todo{Can we put this
into the ASCL?  That would give us a nice record} that creates headers
for inclusion in the FITS files of scanned plates. The header2011
software is tightly related with the Wide-Field Plate Database (WFPDB),
as the software uses the WFPDB files\todo{is this still true?  Is
someone still collecting and ingesting these files?} as a source of
plate metadata.

In this document, we propose a refinement to the header format of Paper
I, closely following the FITS Standard, \citep{std:FITS}, where this
specification was originally written against version 3.0. The
modifications are described in Section~\ref{sect:format}, a complete
sample header is shown in Section~\ref{sect:samplehdr}, and a sample
header created with the header2011 software is given in
Section~\ref{sect:platehdr}.

\section{FITS header format}
\label{sect:format}

For better readability of headers, we propose organizing keyword records
into groups of related keywords and separating the groups with keywords
that have a blank name:

\begin{lstlisting}
KEYWORD1= 'value   '           / sample keyword
KEYWORD2= 'value   '           / sample keyword
        --------------------------------------- Original data of the observation
KEYWORD3= 'value   '           / sample keyword
KEYWORD4= 'value   '           / sample keyword
        ----------------------------------------------------- Photographic plate
KEYWORD5= 'value   '           / sample keyword
KEYWORD6= 'value   '           / sample keyword
        --------------------------------------- Computed data of the observation
KEYWORD7= 'value   '           / sample keyword
KEYWORD8= 'value   '           / sample keyword
\end{lstlisting}

In the next subsections we denote the omitted and missing keywords with
a hyphen (-).

Lowercase n and i in keyword names denote numbering: for example, RAn
becomes RA1, RA2, RA3, etc. Numbers in keywords are not padded with
zeros.

\subsection{Group 1 -- mandatory and array-description keywords}

\begin{inlinetable}
\footnotesize
\begin{tabular}{lllp{0.5\textwidth}}
\sptablerule
\textbf{Keyword}&\textbf{Paper I}&\textbf{Type}&\textbf{Description}\\
\sptablerule
SIMPLE     &SIMPLE     &logical     &(FITS Standard) File conforms to FITS
Standard.\\
BITPIX     &BITPIX     &integer     &(FITS Standard) Number of bits per data
pixel.\\
NAXIS     &NAXIS      &integer     &(FITS Standard) Number of data axes.\\
NAXIS1     &NAXIS1     &integer     &(FITS Standard) Length of data axis 1
(number of pixels in a row).\\
NAXIS2     &NAXIS2     &integer     &(FITS Standard) Length of data axis 2
(number of rows).\\
BSCALE     &BSCALE     &float             &(FITS Standard) For unsigned 16-bit
integer data, the value should be 1.0.\\
BZERO     &BZERO     &integer     &(FITS Standard) For unsigned 16-bit integer
data, the value should be 32768.\\
-     &EXTEND     &logical     &Not required by FITS Standard, even if the FITS
    file does contain extensions.\\
END     &END     &no value     &(FITS Standard) Marks the end of the
header.\\
\end{tabular}
\end{inlinetable}

Main differences with Paper I:

EXTEND → omit

BZERO = 65536 → 32768

Example:

\begin{lstlisting}
SIMPLE  =                    T / file conforms to FITS standard
BITPIX  =                   16 / number of bits per data pixel
NAXIS   =                    2 / number of data axes
NAXIS1  =                18904 / length of data axis 1
NAXIS2  =                18904 / length of data axis 2
BSCALE  =                  1.0 / physical_value = BZERO + BSCALE * array_value
BZERO   =                32768 / physical_value = BZERO + BSCALE * array_value
END
\end{lstlisting}


\subsection{Group 2 -- original data of the observation}

This group of keyword records provides the original information about
the observation, as described in the observation logbook and other
sources.

\begingroup
\footnotesize
\begin{longtable}{lllp{0.5\textwidth}}
\sptablerule
\textbf{Keyword}&\textbf{Paper
I}&\textbf{Type}&\textbf{Description}\\
\sptablerule
\endhead
DATEORIG  &-        &string  &
  Original recorded date of the observation (evening date)\\
TMS-ORIG  &ST       &string  &
  Original recorded time of the start of the observation (format ``TZ
  hh:mm:ss'', where TZ is time zone). Time zone can be 'ST' (sidereal
  time), 'UT' (universal time), or any time zone.  Multiple time
  notations are separated with commas (e.g. 'UT 18:13, ST 02:44').\\
TME-ORIG  &-        &string  &
  Original recorded time of the end of the observation. See TMS-ORIG for details.\\
TIMEFLAG  &-        &string  &
  Quality flag of the recorded observation time: 'error', 'missing', 'uncertain'.\\
RA-ORIG   &RAEPOBS  &string  &
  Original recorded right ascension of the telescope pointing (plate center)\\
DEC-ORIG  &DECEPOBS &string  &
  Original recorded declination of the telescope pointing (plate center)\\
COORFLAG  &-        &string  &
  Quality flag of the recorded coordinates (right ascension and
  declination): 'error', 'missing', 'uncertain'.\\
OBJECT    &OBJECT FIELD &string &
  (FITS Standard) Name of the observed object or field. If there are
  more than one field observed, then the value shall be 'multiple' and
  individual names shall be given with the OBJECTn keywords.\\
OBJTYPE   &-        &string  &
  Object type (literal text), as listed in the WFPDB\\
EXPTIME   &EXPTIME  &float   &
  Exposure time of the first exposure, expressed in seconds\\
NUMEXP    &MULTIEXP &integer &Number of exposures\\
DATEORn   &-        &string  &
  Original recorded date of the n-th exposure (n = 1…99), if exposures
  were made on multiple nights. Not used, when all exposures are from
  one night, given by DATEORIG.\\
TMS-ORn   &-        &string  &
  Original recorded time of the start of the n-th exposure (n = 1…99). See TMS-ORIG for details.\\
TME-ORn   &-        &string  &
  Original recorded time of the end of the n-th exposure (n = 1…99). See TMS-ORIG for details.\\
RA-ORn    &-        &string  &
  Original right ascension of the telescope pointing during the n-th
  exposure (n = 1…99). Not used, if only one pointing was used.\\
DEC-ORn   &-        &string   &
  Original declination of the telescope pointing during the n-th
  exposure (n = 1…99). Not used, if only one pointing was used.\\
OBJECTn   &-        &string  &
    Object (field) name on the n-th exposure (n = 1…99). Not used, if only
    one object (field) was observed.\\
OBJTYPn   &-        &string  &Object type that corresponds to OBJECTn (n
= 1…99)\\
EXPTIMn   &-        &float   & Exposure time of the n-th exposure (n = 1…99)\\
OBSERVAT  &OBSERVAT &string  &Observatory name\\
SITENAME  &-        &string  &
  Observatory site name. Useful if the observatory has more than one
  observing site.\\
SITELONG  &SITELONG &float   &
  East longitude of the observing site, in decimal degrees\\
SITELAT   &SITELAT  &float   &
  Latitude of the observing site, in decimal degrees\\
SITEELEV  &SITEALTI &float   &
  Elevation of the observatory site [m].  Keyword SITEELEV is more
  widely used than SITEALTI.\\
TELESCOP  &TELESCOP &string  &(FITS Standard) Telescope name\\
TELAPER   &TELAPER  &float   &Clear aperture of the telescope [m]\\
TELFOC    &TELFOC   &float   &Focal length of the telescope [m]\\
TELSCALE  &TELSCALE &float   &Plate scale of the telescope [arcsec/mm]\\
INSTRUME  &INSTRUME &string  &(FITS Standard) Instrument name\\
DETNAM    &DETNAM   &string  &Detector name: 'photographic plate'\\
METHOD    &-        &string  &
  Observation method (literal text). A list of possible values is given
  in the WFPDB.\\
FILTER    &FILTER   &string  &Filter type\\
PRISM     &-        &string  &
  Information about the objective prism used\\
PRISMANG  &PRIZMANG &string  &
  Angle of the objective prism (format "deg:min")\\
DISPERS   &DISPERS  &float   &Dispersion [Angstrom/mm]\\
GRATING   &-        &string  &Information about the grating used\\
FOCUS     &-        &float   &
  Focus value (from logbook). Used when a single value is given in the
  logs.\\
FOCUSn    &-        &float   &
  Focus value of the n-th exposure ($n = 1\cdots 99$)\\
TEMPERAT  &-        &float   &Air temperature (from logbook).\\
CALMNESS  &-        &string  &
  Calmness (seeing conditions), scale 1–5 (German: Ruhe)\\
SHARPNES  &-        &string  &Sharpness, scale 1–5 (German: Schärfe)\\
TRANSPAR  &-        &string  &
  Transparency, scale 1–5 (German: Durchsicht, Klarheit)\\
SKYCOND   &-        &string  &Notes on sky conditions (from logbook)\\
OBSERVER  &OBSERVER &string  &(FITS Standard) Observer name\\
OBSNOTES  &-        &string  &Observer notes (from logbook)\\
NOTES     &-        &string  &Miscellaneous notes \\

\end{longtable}
\endgroup

Exposure times are given in seconds. This is different from Paper I and
the WFPDB that specify exposure time in decimal minutes, rounded to the
first decimal place.

In case of multiple exposures (NUMEXP is greater than 1), exposure times
of all sub-exposures can be given with the EXPTIMn keywords, where n is
the exposure number in the range 1…99. The EXPTIME and EXPTIM1 keywords
have the same value.

Example:

\begin{lstlisting}
EXPTIME =                600.0 / [s] exposure time (of exposure 1)
NUMEXP  =                    3 / number of exposures of the plate
EXPTIM1 =                600.0 / [s] exposure time of exposure 1
EXPTIM2 =                 60.0 / [s] exposure time of exposure 2
EXPTIM3 =                  2.0 / [s] exposure time of exposure 3
\end{lstlisting}

In case of a single exposure, EXPTIM1 keyword is omitted:

\begin{lstlisting}
EXPTIME =               1800.0 / [s] exposure time (of exposure 1)
NUMEXP  =                    1 / number of exposures of the plate
\end{lstlisting}

SITELONG and SITELAT are expressed in decimal degrees (rather than a
sexagesimal character string).

Main differences with Paper I:

new keywords DATEORIG, TMS-ORIG, TME-ORIG, TIMEFLAG, and COORFLAG

ST → TMS-ORIG (provided that the original time is given as sidereal time)

RAEPOBS, DECEPOBS → RA-ORIG, DEC-ORIG

FIELD → OBJECT

EXPTIME expressed in minutes → seconds

MULTIEXP → NUMEXP

new keywords DATEORn, TMS-ORn, TME-ORn, RA-ORn, DEC-ORn, OBJECTn, OBJTYPn, EXPTIMn, where n is the exposure number in the range of 1…99

new keywords OBJTYPE and METHOD, based on the WFPDB

new keyword SITENAME

SITELONG, SITELAT expressed in sexagesimal format → decimal degrees

SITEALTI → SITEELEV

PRIZMANG → PRISMANG

new keyword GRATING

new keywords TEMPERAT, SEEING, SHARPNES, TRANSPAR, SKYCOND, OBSNOTES, and NOTES

Example:

\begin{lstlisting}
        --------------------------------------- Original data of the observation
DATEORIG= '1910-08-02'         / recorded date of the observation
TMS-ORIG= 'ST 18:11:16'        / recorded time of the start of the observation
TME-ORIG= '        '           / recorded time of the end of the observation
TIMEFLAG= 'uncertain'          / quality of the recorded time
RA-ORIG = '19:11:42'           / recorded right ascension of telescope pointing
DEC-ORIG= '15:04:00'           / recorded declination of telescope pointing
COORFLAG= 'uncertain'          / quality of the recorded coordinates
OBJECT  = 'SA 87   '           / name of the observed object or field
OBJTYPE = 'field   '           / object type
EXPTIME =               1800.0 / [s] exposure time (of exposure 1)
NUMEXP  =                    1 / number of exposures of the plate
OBSERVAT= 'Astrophysikalische Observatorium Potsdam' / observatory name
SITENAME= 'Potsdam-Telegrafenberg' / observatory site name
SITELONG=            13.064167 / [deg] East longitude of the observatory
SITELAT =            52.380556 / [deg] latitude of the observatory
SITEELEV=                  107 / [m] elevation of the observatory
TELESCOP= 'Zeiss Triplet 15 cm' / telescope name
TELAPER =                 0.15 / [m] clear aperture of the telescope
TELFOC  =                  1.5 / [m] focal length of the telescope
TELSCALE=               137.68 / [arcsec/mm] plate scale of the telescope
INSTRUME= '        '           / instrument
DETNAM  = 'photographic plate' / detector
METHOD  = 'direct photograph'  / method of observation
FILTER  = 'none    '           / filter type
PRISM   = '        '           / objective prism
PRISMANG= '        '           / prism angle "deg:min"
DISPERS =                      / [Angstrom/mm] dispersion
GRATING = '        '           / grating
FOCUS   =                 32.2 / focus value
TEMPERAT=                 21.8 / [deg C] air temperature (degrees Celsius)
CALMNESS= '2-3     '           / sky calmness (scale 1-5)
SHARPNES= '2       '           / sky sharpness (scale 1-5)
TRANSPAR= '1-2     '           / sky transparency (scale 1-5)
SKYCOND = 'moonlight'          / sky conditions
OBSERVER= 'W. Muench'          / observer name
OBSNOTES= 'bad guiding'        / observer notes
NOTES   = 'SA 87 = Kapteyn Selected Area 87' / miscellaneous notes
\end{lstlisting}

Example 2 (multiple time notations):

\begin{lstlisting}
        --------------------------------------- Original data of the observation
DATEORIG= '1964-01-02'         / recorded date of the observation
TMS-ORIG= 'UT 18:13, ST 02:44' / recorded time of the start of exposure 1
TME-ORIG= 'UT 19:13, ST 03:44' / recorded time of the end of exposure 1
TIMEFLAG= '        '           / quality of the recorded time
EXPTIME =               3600.0 / [s] exposure time (of exposure 1)
NUMEXP  =                    1 / number of exposures of the plate
\end{lstlisting}

Example 3 (multiple exposures):

\begin{lstlisting}
        --------------------------------------- Original data of the observation
DATEORIG= '1934-04-01'         / recorded date of the observation
TMS-OR1 = 'ST 10:52'           / recorded time of the start of exposure 1
TMS-OR2 = 'ST 10:54'           / recorded time of the start of exposure 2
TMS-OR3 = 'ST 10:57'           / recorded time of the start of exposure 3
TME-OR1 = 'ST 10:53'           / recorded time of the end of exposure 1
TME-OR2 = 'ST 10:56'           / recorded time of the end of exposure 2
TME-OR3 = 'ST 11:01'           / recorded time of the end of exposure 3
TIMEFLAG= '        '           / quality of the recorded time
RA-ORIG = '        '           / recorded right ascension of telescope pointing
DEC-ORIG= '        '           / recorded declination of telescope pointing
COORFLAG= 'missing '           / quality of the recorded coordinates
OBJECT  = 'RY UMa  '           / name of the observed object or field
OBJTYPE = 'variable star'      / object type
EXPTIME =                 60.0 / [s] exposure time (of exposure 1)
NUMEXP  =                    3 / number of exposures of the plate
EXPTIM1 =                 60.0 / [s] exposure time of exposure 1
EXPTIM2 =                120.0 / [s] exposure time of exposure 2
EXPTIM3 =                240.0 / [s] exposure time of exposure 3
OBSERVAT= 'Astrophysikalische Observatorium Potsdam' / observatory name
SITENAME= 'Potsdam-Telegrafenberg' / observatory site name
SITELONG=            13.064167 / [deg] East longitude of the observatory
SITELAT =            52.380556 / [deg] latitude of the observatory
SITEELEV=                  107 / [m] elevation of the observatory
TELESCOP= 'Zeiss Triplet 15 cm' / telescope name
TELAPER =                 0.15 / [m] clear aperture of the telescope
TELFOC  =                  1.5 / [m] focal length of the telescope
TELSCALE=               137.68 / [arcsec/mm] plate scale of the telescope
INSTRUME= '        '           / instrument
DETNAM  = 'photographic plate' / detector
METHOD  = 'direct photograph, multi-exposure' / method of observation
FILTER  = 'none    '           / filter type
PRISM   = '        '           / objective prism
PRISMANG= '        '           / prism angle "deg:min"
DISPERS =                      / [Angstrom/mm] dispersion
GRATING = '        '           / grating
FOCUS   =                 34.4 / focus value
TEMPERAT=                    8 / [deg C] air temperature (degrees Celsius)
CALMNESS= '        '           / sky calmness (scale 1-5)
SHARPNES= '        '           / sky sharpness (scale 1-5)
TRANSPAR= '        '           / sky transparency (scale 1-5)
SKYCOND = 'clouds  '           / sky conditions
OBSERVER= 'W. Muench'          / observer name
OBSNOTES= 'poor transparency'  / observer notes
NOTES   = '        '           / miscellaneous notes
\end{lstlisting}

\subsection{Group 3 – information about the photographic plate}

\begin{inlinetable}
\footnotesize
\begin{tabular}{lllp{0.5\textwidth}}
\sptablerule
\textbf{Keyword}&\textbf{Paper I}&\textbf{Type}&\textbf{Description}\\
\sptablerule
PLATENUM &PLATENUM &string &Plate number in original observation
catalogue\\
WFPDB-ID &PLATE-ID &string &Plate identification in the WFPDB\\
SERIES   &  -      &string &Series or survey in which the plate belongs,
e.g. Carte du Ciel, Kapteyn Selected Areas, etc.\\
PLATEFMT &-        &string &Plate format (e.g. '9x12', '20x20')\\
PLATESZ1 &PLATESZ  &float  &Plate size along axis 1\\
PLATESZ2 &PLATESZ  &float  &Plate size along axis 2\\
FOV1     &CUNIT1   &float  &Field of view along axis 1\\
FOV2     &CUNIT2   &float  &ield of view along axis 2\\
EMULSION &EMULSION &string &Type of the photographic emulsion\\
-        &COLOR    &N/A    &The of this keyword is not explained in Paper
         I.\\
DEVELOP  &-        &string &Plate development information (developer,
time)\\
PQUALITY &PQUALITY &string &Quality of the plate\\
PLATNOTE &-        &string &Notes about the plate (e.g. contact copy of
the original plate)\\
\end{tabular}
\end{inlinetable}

We propose giving the plate size separately for both axes, thus replacing the PLATESZ keyword (character string) with PLATESZ1 and PLATESZ2 (floating-point numbers).

We specify the field of view along both axes with the FOV1 and FOV2 keywords, replacing CUNIT1 and CUNIT2 that are reserved for the WCS.

Main differences with Paper I:

PLATE-ID → WFPDB-ID

new keyword SERIES

PLATESZ → PLATESZ1, PLATESZ2

CUNIT1, CUNIT2 → FOV1, FOV2

COLOR → omit

new keyword PLATNOTE


Example:

\begin{lstlisting}
        ----------------------------------------------------- Photographic plate
PLATENUM= '317     '           / plate number in original observation catalogue
WFPDB-ID= 'POT015_000317'      / plate identification in the WFPDB
SERIES  = 'Kapteyn Selected Areas' / plate series
PLATEFMT= '20x20   '           / plate format in cm
PLATESZ1=                 20.0 / [cm] plate size along axis 1
PLATESZ2=                 20.0 / [cm] plate size along axis 2
FOV1    =                 7.65 / [deg] field of view along axis 1
FOV2    =                 7.65 / [deg] field of view along axis 2
EMULSION= 'Schleussner'        / photographic emulsion type
DEVELOP = '        '           / plate development information
PQUALITY= 'broken  '           / quality of plate
PLATNOTE= 'contact copy of original plate that is not available' / plate notes
\end{lstlisting}

\subsection{Group 4 – computed data of the observation}

In this group of keyword records, we provide data of the observation
that are computed on the basis of the original data.

\begingroup
\footnotesize
\begin{longtable}{lllp{0.5\textwidth}}
\sptablerule
\textbf{Keyword}&\textbf{Paper I}&\textbf{Type}&\textbf{Description}\\
\sptablerule
DATE-OBS  &DATE-OBST &\\
          &TIME-OBS  &string     &(FITS Standard) UT date and time of the
          start of the observation (format "YYYY-MM-DDThh:mm:ss", or
          "YYYY-MM-DD" if time is not specified). The date may differ
          from DATEORIG, because the original date usually refers to the
          evening of the observing night.\\
DT-OBSn   &-         &string     &UT date and time of the start of the
n-th exposure (n = 1…99)\\
DATE-AVG  &UT        &string     &(FITS Standard) UT date and time of the
mid-point of the first exposure (format "YYYY-MM-DDThh:mm:ss")\\
DT-AVGn   &-         &string     &UT date and time of the mid-point of
the n-th exposure (n = 1…99)\\
DATE-END  &DATE-OBS  &\\
          &TIME-END  &string     &UT date and time of the end of the
          first exposure (format "YYYY-MM-DDThh:mm:ss")\\
DT-ENDn   &-         &string     &UT date and time of the end of the n-th
exposure (n = 1…99)\\
YEAR      &-         &float     &Decimal year of the start of the first
exposure\\
YEARn     &-         &float     &Decimal year of the start of the n-th
exposure (n = 1…99)\\
YEAR-AVG  &EPOCH     &float     &Decimal year of the mid-point of the
first exposure\\
YR-AVGn   &-         &float     &Decimal year of the mid-point of the
n-th exposure (n = 1…99)\\
JD        &-         &float     &Julian date at the start of exposure 1\\
JDn       &-         &float     &Julian date at the start of the n-th
exposure (n = 1…99)\\
JD-AVG    &JD        &float     &Julian date at the mid-point of the
first exposure\\
JD-AVGn   &-         &float     &Julian date at the mid-point of the n-th
exposure (n = 1…99)\\
HJD-AVG   &-         &float     &Heliocentric Julian date at the
mid-point of the first exposure\\
HJD-AVn   &-         &float     &Heliocentric Julian date at the
mid-point of the n-th exposure (n = 1…99)\\
RA        &RA        &string    & Right ascension of the telescope
pointing (equinox J2000, sexagesimal format "h:m:s")\\
DEC       &DEC       &string    & Declination of the telescope pointing
(equinox J2000, sexagesimal format "d:m:s")\\
RAn       &-         &string    & Right ascension of the telescope
pointing, n-th exposure=(n = 1…99). Used only when different fields were
exposed on the same plate.\\
DECn      &-         &string    & Declination of the telescope pointing,
n-th exposure (n = 1…99). Used only when different fields were exposed
on the same plate.\\
RA\_DEG    &-         &float    & Right ascension of the telescope
pointing in decimal degrees (equinox J2000)\\
DEC\_DEG   &-         &float    & Declination of the telescope pointing in
decimal degrees (equinox J2000)\\
RA\_DEGn   &-         &float    & Right ascension of the telescope
pointing in decimal degrees, n-th exposure (n = 1…99). Used only when
different fields were exposed on the same plate.\\
DEC\_DEn   &-         &float    & Declination of the telescope pointing in
decimal degrees, n-th exposure (n = 1…99). Used only when different
fields were exposed on the same plate. \\
\end{longtable}
\endgroup


We replace the EPOCH keyword (Paper I) with YEAR-AVG. The EPOCH keyword
is deprecated in the FITS Standard. It was previously used to give the
equinox in years for the celestial coordinate system in which positions
were expressed. We reserve the EQUINOX keyword for the World Coordinate
System, as required by the FITS Standard.

The RA, DEC, RA\_DEG, and DEC\_DEG keywords provide the precessed
coordinates of the original recorded coordinates to the equinox J2000.

Main differences with Paper I:

DATE-OBS, TIME-OBS → DATE-OBS

TIME-END → DATE-END

UT → DATE-AVG

date format "YYYY-MM-DD hh:mm:ss" → "YYYY-MM-DDThh:mm:ss"

EPOCH → YEAR-AVG

JD → JD-AVG

new keywords YEAR, HJD-AVG

new keywords RA\_DEG, DEC\_DEG

numbered keywords for multiple exposures

EQUINOX → if necessary, specify with the WCS keywords

Example:

\begin{lstlisting}
        --------------------------------------- Computed data of the observation
DATE-OBS= '1910-08-02T22:21:01' / UT date of the start of the observation
DATE-AVG= '1910-08-02T22:36:01' / UT date of the mid-point of exposure 1
DATE-END= '1910-08-02T22:51:01' / UT date of the end of exposure 1
YEAR    =       1910.583561644 / decimal year of the start of exposure 1
YEAR-AVG=       1910.583561644 / decimal year of the mid-point of exposure 1
JD      =       2418886.441678 / Julian date at the start of exposure 1
JD-AVG  =       2418886.441678 / Julian date at the mid-point of exposure 1
HJD-AVG =       2418886.441678 / heliocentric JD at the mid-point of exposure 1
RA      = '19:15:48'           / right ascension of pointing (J2000) "h:m:s"
DEC     = '+15:13:20'          / declination of pointing (J2000) "d:m:s"
RA_DEG  =           288.950000 / [deg] right ascension of pointing (J2000)
DEC_DEG =            15.222222 / [deg] declination of pointing (J2000)
\end{lstlisting}

Example 2:

\begin{lstlisting}
        --------------------------------------- Computed data of the observation
DATE-OBS= '1934-01-25T20:36:56' / UT date of the start of exposure 1
DT-OBS1 = '1934-01-25T20:36:56' / UT date of the start of exposure 1
DT-OBS2 = '1934-01-25T20:45:55' / UT date of the start of exposure 2
DT-OBS3 = '1934-01-25T20:55:53' / UT date of the start of exposure 3
DT-OBS4 = '1934-01-25T20:57:53' / UT date of the start of exposure 4
DATE-AVG= '1934-01-25T20:40:56' / UT date of the mid-point of exposure 1
DT-AVG1 = '1934-01-25T20:40:56' / UT date of the mid-point of exposure 1
DT-AVG2 = '1934-01-25T20:48:25' / UT date of the mid-point of exposure 2
DT-AVG3 = '1934-01-25T20:56:23' / UT date of the mid-point of exposure 3
DT-AVG4 = '1934-01-25T20:58:53' / UT date of the mid-point of exposure 4
DATE-END= '1934-01-25T20:44:55' / UT date of the end of exposure 1
DT-END1 = '1934-01-25T20:44:55' / UT date of the end of exposure 1
DT-END2 = '1934-01-25T20:50:54' / UT date of the end of exposure 2
DT-END3 = '1934-01-25T20:56:53' / UT date of the end of exposure 3
DT-END4 = '1934-01-25T20:59:52' / UT date of the end of exposure 4
YEAR    =        1934.06806018 / decimal year of the start of exposure 1
YEAR1   =        1934.06806018 / decimal year of the start of exposure 1
YEAR2   =        1934.06807726 / decimal year of the start of exposure 2
YEAR3   =        1934.06809621 / decimal year of the start of exposure 3
YEAR4   =        1934.06810001 / decimal year of the start of exposure 4
YEAR-AVG=        1934.06806779 / decimal year of the mid-point of exposure 1
YR-AVG1 =        1934.06806779 / decimal year of the mid-point of exposure 1
YR-AVG2 =        1934.06808202 / decimal year of the mid-point of exposure 2
YR-AVG3 =        1934.06809716 / decimal year of the mid-point of exposure 3
YR-AVG4 =        1934.06810192 / decimal year of the mid-point of exposure 4
JD      =        2427463.35898 / Julian date at the start of exposure 1
JD1     =        2427463.35898 / Julian date at the start of exposure 1
JD2     =        2427463.36522 / Julian date at the start of exposure 2
JD3     =        2427463.37214 / Julian date at the start of exposure 3
JD4     =        2427463.37353 / Julian date at the start of exposure 4
JD-AVG  =        2427463.36176 / Julian date at the mid-point of exposure 1
JD-AVG1 =        2427463.36176 / Julian date at the mid-point of exposure 1
JD-AVG2 =        2427463.36696 / Julian date at the mid-point of exposure 2
JD-AVG3 =        2427463.37249 / Julian date at the mid-point of exposure 3
JD-AVG4 =        2427463.37422 / Julian date at the mid-point of exposure 4
HJD-AVG =                      / heliocentric JD at the mid-point of exposure 1
RA      = '        '           / right ascension of pointing (J2000) "h:m:s"
DEC     = '        '           / declination of pointing (J2000) "d:m:s"
RA_DEG  =                      / [deg] right ascension of pointing (J2000)
DEC_DEG =                      / [deg] declination of pointing (J2000)
\end{lstlisting}


\subsection{Group 5 – scan details}

This group contains information about scanner hardware and software
settings, also the name of the scan author and the date of the scan.

\begin{inlinetable}
\footnotesize
\begin{tabular}{lllp{0.5\textwidth}}
\sptablerule
\textbf{Keyword}&\textbf{Paper I}&\textbf{Type}&\textbf{Description}\\
\sptablerule
SCANRES1 &SCANRES    &integer &Scan resolution along axis 1 [dpi]\\
SCANRES2 &SCANRES    &integer &Scan resolution along axis 2 [dpi]\\
PIXSIZE1 &XPIXELSZ   &float   &Pixel size along axis 1 [$\mu\rm m$]\\
PIXSIZE2 &YPIXELSZ   &float   &Pixel size along axis 2 [$\mu\rm m$]\\
SCANSOFT &-          &string  &Name of the scanning software\\
-        &SCANHCUT   &        &Omit: scan high-cut value\\
-        &SCANLCUT   &        &Omit: scan low-cut value\\
SCANGAM  &SCANGAM    &float   &Scan gamma value\\
SCANFOC  &SCANFOC    &string  &Scan focus (e.g. 'glass')\\
WEDGE    &WEDGE      &string  &Type of photometric step-wedge\\
DATESCAN &DATE-SCN   &string  &
  Scan date and time (UTC, format "YYYY-MM-DDThh:mm:ss")\\
SCANAUTH &AUTHOR     &string  &Author of the scan\\
SCANNOTE &-          &string  &
  Notes about the scan (e.g. scan orientation)\\
-        &REFERENC   &        &We propose omitting this keyword \
\end{tabular}
\end{inlinetable}


We propose replacing the SCANRES keyword with the SCANRES1 and SCANRES2
keywords, indicating scan resolution along both image axes separately.

By FITS Standard, the AUTHOR and REFERENCE keywords are used when the
data in the FITS file were compiled from a publication or multiple
sources. For digitized photographic plates, these keywords are not
appropriate for specifying the author of the scan nor acknowledging any
funding sources. We propose replacing the AUTHOR keyword with SCANAUTH
and providing acknowledgments with the COMMENT keyword (Group 8).

Main differences with Paper I:

SCANRES → SCANRES1, SCANRES2

XPIXELSZ → PIXSIZE1

YPIXELSZ → PIXSIZE2

PIXSIZE1, PIXSIZE2 unit: [microns] → [um]

new keywords SCANSOFT, SCANNOTE

SCANHCUT, SCANLCUT → omit

SCANFOC type: float → string

date format "YYYY-MM-DD hh:mm:ss" → "YYYY-MM-DDThh:mm:ss"

AUTHOR → SCANAUTH

REFERENC → COMMENT

Example:

\begin{lstlisting}
        ------------------------------------------------------------------- Scan
SCANNER = 'Epson Expression 10000XL' / scanner name
SCANRES1=                 2400 / [dpi] scan resolution along axis 1
SCANRES2=                 2400 / [dpi] scan resolution along axis 2
PIXSIZE1=              10.5833 / [um] pixel size along axis 1
PIXSIZE2=              10.5833 / [um] pixel size along axis 2
SCANSOFT= 'VueScan '           / name of the scanning software
SCANGAM =                  1.0 / scan gamma value
SCANFOC = 'glass'              / scan focus
WEDGE   = 'Danes-Picta TG21S'  / type of photometric step-wedge
DATESCAN= '2011-05-17T10:33:26' / scan date and time
SCANAUTH= 'K. Tsvetkova'       / author of scan
\end{lstlisting}


\subsection{Group 6 – data files}

\begin{inlinetable}
\footnotesize
\begin{tabular}{lllp{0.5\textwidth}}
\sptablerule
\textbf{Keyword}&\textbf{Paper I}&\textbf{Type}&\textbf{Description}\\
\sptablerule
FILENAME &FILENAME &string    &Filename of the plate scan (this file)\\
FN-SCNi  &-        &string    &
  Filename of the i-th scan of the same plate (i = 1…99)\\
FN-WEDGE &-        &string    &Filename of the wedge scan\\
FN-PRE   &-        &string    &
  Filename of the preview image (annotated plate)\\
FN-COVER &-        &string    &
  Filename of the plate cover (envelope) image\\
FN-LOGB  &-        &string    &Filename of the logbook image\\
FN-NTBi  &-        &string    &
  Filename of the i-th notebook image (i = $1\cdots 99$)\\
-        &URL      &          &We propose omitting this keyword.\\
ORIGIN   &ORIGIN   &string    &
  (FITS Standard) Institute responsible for creating the FITS file\\
DATE     &DATE     &string    &
  (FITS Standard) Date and time of the last change of the file \\
\end{tabular}
\end{inlinetable}

Main differences with Paper I:

new keywords FN-SCNi, FN-WEDGE, FN-PRE, FN-COVER, FN-LOGB, and FN-NTBi

URL → omit

Example:

\begin{lstlisting}
        ------------------------------------------------------------- Data files
FILENAME= 'POT015_000317.fits' / filename of this file
FN-WEDGE= 'POT015_000317w.fits' / filename of the wedge scan
FN-PRE  = 'POT015_000317_pre.jpg' / filename of the preview image
FN-COVER= '        '           / filename of the plate cover image
FN-LOGB = 'POT015_000317-000334.jpg' / filename of logbook image
ORIGIN  = 'Leibniz-Institut fuer Astrophysik Potsdam (AIP)'
DATE    = '2013-04-09T12:00:00' / last change of this file
\end{lstlisting}


Example 2:

\begin{lstlisting}
        ------------------------------------------------------------- Data files
FILENAME= 'LA00508x.fits'      / filename of this file
FN-SCN1 = 'LA00508x.fits'      / filename of scan 1
FN-SCN2 = 'LA00508y.fits'      / filename of scan 2
FN-WEDGE= '        '           / filename of the wedge scan
FN-PRE  = 'LA00508_pre.jpg'    / filename of the preview image
FN-COVER= 'LA00508_cover.jpg'  / filename of the plate cover image
FN-LOGB = 'LA-PV01-LA00501_00510.jpg' / filename of logbook image
FN-NTB1 = 'LA-LB04-1916-10-18a.jpg' / filename of notebook image 1
FN-NTB2 = 'LA-LB04-1916-10-18b.jpg' / filename of notebook image 2
FN-NTB3 = 'LA-LB04-1916-10-18c.jpg' / filename of notebook image 3
FN-NTB4 = 'LA-LB04-1916-10-18d.jpg' / filename of notebook image 4
FN-NTB5 = 'LA-LB04-1916-10-18e.jpg' / filename of notebook image 5
FN-NTB6 = 'LA-LB04-1916-10-18f.jpg' / filename of notebook image 6
ORIGIN  = 'Hamburger Sternwarte' /
DATE    = '2013-12-12T13:42:00' / last change of this file
\end{lstlisting}

\subsection{Group 7 – World Coordinate System (WCS)}

The astrometric information are given with the World Coordinate System
(WCS) keywords, as described in the FITS Standard.

If the EQUINOX keyword is not given, then coordinates are assumed to
refer to the International Celestial Reference System (ICRS).

When the WCS solution is not possible through matching stars in the
scan, the WCS keywords can be used to provide the approximate
coordinates, based on the plate scale and the precessed coordinates of
the original telescope pointing.

Example:

\begin{lstlisting}
        -------------- World Coordinate System (WCS)
WCSAXES = 2 / number of axes in the WCS description
RADESYS = 'FK5'  / name of the reference frame
EQUINOX = 2000.0 / epoch of the mean equator and equinox in years
CTYPE1 = 'RA-TAN' / TAN (gnomonic) projection
CTYPE2 = 'DEC-TAN' / TAN (gnomonic) projection
CUNIT1 = 'deg' / physical units of CRVAL and CDELT for axis 1
CUNIT2 = 'deg' / physical units of CRVAL and CDELT for axis 2
CRPIX1 = 9452.5 / reference pixel for axis 1
CRPIX2 = 9452.5 / reference pixel for axis 2
CRVAL1 = 288.95 / right ascension at the reference point
CRVAL2 = 15.222222 / declination at the reference point
CD1_1 = -0.0004047524 / transformation matrix
CD1_2 = 0.0 / transformation matrix
CD2_1 = 0.0 / transformation matrix
CD2_2 = 0.0004047524 / transformation matrix
LONPOLE = 0.0 / native longitude of the celestial pole
\end{lstlisting}


\subsection{Group 8 – modification history and acknowledgements}

Modification history is given with the HISTORY keyword.

Comments and acknowledgements are given with the COMMENT keyword.

Example:

\begin{lstlisting}
        --------------------------------------------------- Modification history
HISTORY Header written with PyPlates at 2013-12-18T12:00:00
HISTORY WCS modified by T. Tuvikene (AIP) at 2013-12-19T12:00:00
        ------------------------------------------------------- Acknowledgements
COMMENT The digitization of this plate was funded by the German Research
COMMENT Foundation (DFG) grant STE: 710/6-1,20.11.2009 and partially by the
COMMENT grants of the Bulgarian Ministry of Education and Science
COMMENT DO-02-273/275,18.12.2009. The 2011 May-June stay of K. Tsvetkova at AIP
COMMENT was funded by DO-02-275 MON.
COMMENT
COMMENT Publications based on this digitized photographic plate are requested to
COMMENT include the following acknowledgement.
COMMENT
COMMENT Based on photographic data of the Leibniz-Institut fuer Astrophysik
COMMENT Potsdam (AIP). The Kapteyn Selected Areas Survey was obtained with the
COMMENT 80-cm Great Refractor and the 15-cm Zeiss Triplet telescope at
COMMENT Potsdam-Telegrafenberg in 1910-1933. The project of plate digitization
COMMENT was funded by the grants of the German Research Foundation (DFG) and the
COMMENT Bulgarian Ministry of Education and Science.
\end{lstlisting}


\section{Complete sample header}
\label{sect:samplehdr}

\begin{lstlisting}
SIMPLE  =                    T / file conforms to FITS standard
BITPIX  =                   16 / number of bits per data pixel
NAXIS   =                    2 / number of data axes
NAXIS1  =                18904 / length of data axis 1
NAXIS2  =                18904 / length of data axis 2
BSCALE  =                  1.0 / physical_value = BZERO + BSCALE * array_value
BZERO   =                32768 / physical_value = BZERO + BSCALE * array_value
        --------------------------------------- Original data of the observation
DATEORIG= '1910-08-02'         / recorded date of the observation
TMS-ORIG= 'ST 18:11:16'        / recorded time of the start of the observation
TME-ORIG= '        '           / recorded time of the end of the observation
TIMEFLAG= 'uncertain'          / quality of the recorded time
RA-ORIG = '19:11:42'           / recorded right ascension of telescope pointing
DEC-ORIG= '15:04:00'           / recorded declination of telescope pointing
COORFLAG= 'uncertain'          / quality of the recorded coordinates
OBJECT  = 'SA 87   '           / name of the observed object or field
OBJTYPE = 'field   '           / object type
EXPTIME =               1800.0 / [s] exposure time (of exposure 1)
NUMEXP  =                    1 / number of exposures of the plate
OBSERVAT= 'Astrophysikalische Observatorium Potsdam' / observatory name
SITENAME= 'Potsdam-Telegrafenberg' / observatory site name
SITELONG=            13.064167 / [deg] East longitude of the observatory
SITELAT =            52.380556 / [deg] latitude of the observatory
SITEELEV=                  107 / [m] elevation of the observatory
TELESCOP= 'Zeiss Triplet 15 cm' / telescope name
TELAPER =                 0.15 / [m] clear aperture of the telescope
TELFOC  =                  1.5 / [m] focal length of the telescope
TELSCALE=               137.68 / [arcsec/mm] plate scale of the telescope
INSTRUME= '        '           / instrument
DETNAM  = 'photographic plate' / detector
METHOD  = 'direct photograph'  / method of observation
FILTER  = 'none    '           / filter type
PRISM   = '        '           / objective prism
PRISMANG= '        '           / prism angle "deg:min"
DISPERS =                      / [Angstrom/mm] dispersion
GRATING = '        '           / grating
FOCUS   =                 32.2 / focus value
TEMPERAT=                 21.8 / [deg C] air temperature (degrees Celsius)
CALMNESS= '2-3     '           / sky calmness (scale 1-5)
SHARPNES= '2       '           / sky sharpness (scale 1-5)
TRANSPAR= '1-2     '           / sky transparency (scale 1-5)
SKYCOND = 'moonlight'          / sky conditions
OBSERVER= 'W. Muench'          / observer name
OBSNOTES= 'bad guiding'        / observer notes
NOTES   = 'SA 87 = Kapteyn Selected Area 87' / miscellaneous notes
        ----------------------------------------------------- Photographic plate
PLATENUM= '317     '           / plate number in original observation catalogue
WFPDB-ID= 'POT015_000317'      / plate identification in the WFPDB
SERIES  = 'Kapteyn Selected Areas' / plate series
PLATEFMT= '20x20   '           / plate format in cm
PLATESZ1=                 20.0 / [cm] plate size along axis 1
PLATESZ2=                 20.0 / [cm] plate size along axis 2
FOV1    =                 7.65 / [deg] field of view along axis 1
FOV2    =                 7.65 / [deg] field of view along axis 2
EMULSION= 'Schleussner'        / photographic emulsion type
DEVELOP = '        '           / plate development information
PQUALITY= 'broken  '           / quality of plate
PLATNOTE= 'contact copy of original plate that is not available' / plate notes
        --------------------------------------- Computed data of the observation
DATE-OBS= '1910-08-02T22:21:01' / UT date of the start of the observation
DATE-AVG= '1910-08-02T22:36:01' / UT date of the mid-point of exposure 1
DATE-END= '1910-08-02T22:51:01' / UT date of the end of exposure 1
YEAR    =       1910.583561644 / decimal year of the start of exposure 1
YEAR-AVG=       1910.583561644 / decimal year of the mid-point of exposure 1
JD-AVG  =       2418886.441678 / Julian date at the mid-point of exposure 1
HJD-AVG =       2418886.441678 / heliocentric JD at the mid-point of exposure 1 
RA      = '19:15:48'           / right ascension of pointing (J2000) "h:m:s"
DEC     = '+15:13:20'          / declination of pointing (J2000) "d:m:s"
RA_DEG  =           288.950000 / [deg] right ascension of pointing (J2000)
DEC_DEG =            15.222222 / [deg] declination of pointing (J2000)
        ------------------------------------------------------------------- Scan
SCANNER = 'Epson Expression 10000XL' / scanner name
SCANRES1=                 2400 / [dpi] scan resolution along axis 1
SCANRES2=                 2400 / [dpi] scan resolution along axis 2
PIXSIZE1=              10.5833 / [um] pixel size along axis 1
PIXSIZE2=              10.5833 / [um] pixel size along axis 2
SCANSOFT= 'VueScan '           / name of the scanning software
SCANGAM =                  1.0 / scan gamma value
SCANFOC = 'glass'              / scan focus
WEDGE   = 'Danes-Picta TG21S'  / type of photometric step-wedge
DATESCAN= '2011-05-17T10:33:26' / scan date and time
SCANAUTH= 'K. Tsvetkova'       / author of scan
        ------------------------------------------------------------- Data files
FILENAME= 'POT015_000317.fits' / filename of this file
FN-WEDGE= 'POT015_000317w.fits' / filename of the wedge scan
FN-PRE  = 'POT015_000317_pre.jpg' / filename of the preview image
FN-COVER= '        '           / filename of the plate cover image
FN-LOGB = 'POT015_000317-000334.jpg' / filename of logbook image
ORIGIN  = 'Leibniz-Institut fuer Astrophysik Potsdam (AIP)'
DATE    = '2013-04-09T12:00:00' / last change of this file
        ------------------------------------------ World Coordinate System (WCS)
WCSAXES =                    2 / number of axes in the WCS description
RADESYS = 'FK5     '           / name of the reference frame
EQUINOX =               2000.0 / epoch of the mean equator and equinox in years
CTYPE1  = 'RA---TAN'           / TAN (gnomonic) projection
CTYPE2  = 'DEC--TAN'           / TAN (gnomonic) projection
CUNIT1  = 'deg     '           / physical units of CRVAL and CDELT for axis 1
CUNIT2  = 'deg     '           / physical units of CRVAL and CDELT for axis 2
CRPIX1  =               9452.5 / reference pixel for axis 1
CRPIX2  =               9452.5 / reference pixel for axis 2
CRVAL1  =               288.95 / right ascension at the reference point
CRVAL2  =            15.222222 / declination at the reference point
CD1_1   =        -0.0004047524 / transformation matrix
CD1_2   =                  0.0 / transformation matrix
CD2_1   =                  0.0 / transformation matrix
CD2_2   =         0.0004047524 / transformation matrix
LONPOLE =                  0.0 / native longitude of the celestial pole
        --------------------------------------------------- Modification history
HISTORY Header written with PyPlates at 2013-12-18T12:00:00
HISTORY WCS modified by T. Tuvikene (AIP) at 2013-12-19T12:00:00
        ------------------------------------------------------- Acknowledgements
COMMENT The digitization of this plate was funded by the German Research
COMMENT Foundation (DFG) grant STE: 710/6-1,20.11.2009 and partially by the
COMMENT grants of the Bulgarian Ministry of Education and Science
COMMENT DO-02-273/275,18.12.2009. The 2011 May-June stay of K. Tsvetkova at AIP
COMMENT was funded by DO-02-275 MON.
COMMENT
COMMENT Publications based on this digitized photographic plate are requested to
COMMENT include the following acknowledgement.
COMMENT
COMMENT Based on photographic data of the Leibniz-Institut fuer Astrophysik
COMMENT Potsdam (AIP). The Kapteyn Selected Areas Survey was obtained with the
COMMENT 80-cm Great Refractor and the 15-cm Zeiss Triplet telescope at
COMMENT Potsdam-Telegrafenberg in 1910-1933. The project of plate digitization
COMMENT was funded by the grants of the German Research Foundation (DFG) and the
COMMENT Bulgarian Ministry of Education and Science.
END
\end{lstlisting}


\section{Example header created with header2011}

\label{sect:platehdr}

\begin{lstlisting}
SIMPLE  =                  T / file does conform to FITS standard
BITPIX  =                 16 / number of bits per data pixel
NAXIS   =                  2 / number of data axes
NAXIS1  =              18904 / length of data axis 1
NAXIS2  =              18904 / length of data axis 2
EXTEND  =                  T / FITS dataset may contain extensions
BZERO   =              65536 /
BSCALE  =                  1 /
INVERTED=                  T / T -  big-endian, F - little-endian
DATE    = '2011-06-30 10:53:40' / last change of file
FILENAME= 'POT015_000317.fits' / source file name
PLATENUM= '317             ' / in original observation catalogue
PLATE-ID= 'POT015_000317   ' / WFPDB ident. of plate
OBJECT  = 'SA 87           ' / field name and/or star name
EMULSION= '                ' / photoemulsion type
EXPTIME = 3.000000000000E+01 / exposure time [minutes]
DISPERS =                    / dispersion [A/mm]
MULTIEXP=                  1 / number of exposure of the plate
PQUALITY= '                ' / quality of plate
DATE-OBS= '1910-08-02      ' / date of observation
RA      = '19:15:48        ' / center of plate FK5
DEC     = '+15:13:20       ' / center of plate FK5
EQUINOX = 2.000000000000E+03 / equatorial coordinates definition
UT      = '1910-08-02 22:36:01' / date and UT at mean epoch
ST      = '18:11:16        ' / ST at start of the observation
JD      = 2.418886441678E+06 / JD at mean epoch
TIME-OBS= '22:21:01        ' / UT at start of observation
TIME-END= '22:51:01        ' / UT at end of observation
RAEPOBS = '                ' / center of plate at epoch of observation
DECEPOBS= '                ' / center of plate at epoch of observation
EPOCH   = 1.910583561644E+03 / epoch of plate
PLATESZ = '20x20           ' / plate size [cm]x[cm]
CUNIT1  = 7.650000000000E+00 / X field size [deg]
CUNIT2  = 7.650000000000E+00 / Y field size [deg]
DETNAM  = 'Photographic Plate' /
OBSERVER= 'W.Muench        ' / observer name
OBSERVAT= 'AO Potsdam      ' / observatory name
INSTRUME= '                ' /
TELESCOP= 'Zeiss Triplet 15 cm Potsdam-Telegrafenberg' / telescop name
SITELONG= '-13:03:51.0     ' / longitude of the obsrvatory
SITELAT = '+52:22:50.0     ' / latitude of the observatory
SITEALTI=                107 / altitude of the observatory
COLOR   = 'Pg              ' /
FILTER  = 'NO              ' / filter type
PRIZMANG= 'NO              ' / prism angle
TELAPER = 1.500000000000E-01 / clear aperture [m]
TELFOC  = 1.500000000000E+00 / focal length [m]
TELSCALE= 1.376800000000E+02 / telescope scale  [arcsec/mm]
SCANNER = 'EPSON EXPRESSION 10000XL' / scanner name
SCANRES =               2400 / scan resolution
XPIXELSZ= 1.058330000000E+01 / X pixel size [microns]
YPIXELSZ= 1.058330000000E+01 / Y pixel size [microns]
SCANHCUT=                255 / focal length [m]
SCANLCUT=                  0 / scan shadow value
SCANGAM = 1.000000000000E+00 / scan gamma value
SCANFOC = 0.000000000000E+00 / scan focus
DATE-SCN= '2011-05-17 10:33:26' / scan date and time
AUTHOR  = 'K. Tsvetkova    ' / author of scan
ORIGIN  = 'Leibniz IAP - WFPDB - Sofia' /
REFERENC= 'May-June DFG Stay of K.Tsvetkova  in AIP, DO-02-275 MON' / reference
URL     = 'vo.aip.de, www.wfpdb.org' / base URL of VO Service to retrieve data
COMMENT
       Based on photographic data of the Leibniz Astrophysical Observatory
       Potsdam - Kapteyn Selected Areas Survey obtained using the 80 cm Great
       Refractor and 15 cm Zeiss Triplet telescope at Telegrafenberg - Potsdam
       in the period 1910-1933. The plates were digitized using professional
       flatbed scanners EPSON 10000XL/V700 and processed in the present digital
       form. The project of Plate digitization was funded by a German DFG grant
       STE: 710/6-1,20.11.2009 and partially of the grants of Bulgarian
       Ministry of Education and Science DO-02-273/275,18.12.2009.

       Investigators using these scans are kindly requested to include the
       above acknowledgements in any publications.

\end{lstlisting}

\appendix
\section{Changes from Previous Versions}

No previous versions yet.  
% these would be subsections "Changes from v. WD-..."
% Use itemize environments.


\bibliography{ivoatex/ivoabib,ivoatex/docrepo,local}


\end{document}
